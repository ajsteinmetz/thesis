%This is where the body of your abstract goes, limited to 150 words
%for a thesis, and 350 words for a dissertation or document.  The
%word count limits apply to the regular Abstract in the thesis and
%to the separate Special Abstract.  Use the same text for both; just
%adjust the margins and heading.  The abstract should summarize your
%work.  The UMI booklet listed in the resources section of the U of
%A manual provides some writing tips.  The abstract for a dissertation
%or document may be longer than one page; word count is more important
%than page length in this section.

%If you are doing a paper submission, submit one copy of the special 
%abstract, and two extra copies of your title page, in the box with 
%the final copies of your thesis.  If you are doing an electronic
%submission, you can ignore the special abstract.

The goal of this thesis is to explore the rich subject of magnetism, spin and dipole moments in physics from both a quantum and classical mechanical perspective. We explore distinct models of magnetic dipoles at both small scale atomic systems to cosmic scales involving the universe as a whole. We comment on the singular nature of the gyromagnetic ratio $g$ which has a natural value of two.

In relativistic quantum mechanics, we investigate wave equations which have distinguishable magnetic behavior and primarily compare the Dirac-Pauli (DP) equation to the Klein-Gordon-Pauli (KGP) equation which are two differing descriptions of fermions. We analyze the homogeneous magnetic field case and the Coulomb problem for hydrogen-like atoms with emphasis on the role of the anomalous magnetic moment (AMM). We further propose extensions to the KGP equation which couple mass and the magnetic moment. Classically, we propose a relativistic covariant model of the Stern-Gerlach force via the introduction of a magnetic four-potential. This model unites the Amperian and Gilbertian models for magnetic dipole moments.

We further apply our ideas to the neutrino sector specifically analyzing the relationship between charge-parity (CP) violation and strong electromagnetic fields. We find the Jarlskog invariant, which characterizes the size of CP violation, to be amplified and/or generated in strong fields. We also show that electromagnetic fields have an effect on neutrino oscillation.

Finally we cover spin polarization of the early universe during the dense electron-positron plasma epoch. A model of spin polarization thermodynamics is proposed. We analyze the paramagnetic characteristics of the plasma when exposed to an external primordial field. We find that only a small spin polarization asymmetry is required to generate field strengths in agreement with those measured today.