%This is where the body of your abstract goes, limited to 150 words
%for a thesis, and 350 words for a dissertation or document.  The
%word count limits apply to the regular Abstract in the thesis and
%to the separate Special Abstract.  Use the same text for both; just
%adjust the margins and heading.  The abstract should summarize your
%work.  The UMI booklet listed in the resources section of the U of
%A manual provides some writing tips.  The abstract for a dissertation
%or document may be longer than one page; word count is more important
%than page length in this section.

%If you are doing a paper submission, submit one copy of the special 
%abstract, and two extra copies of your title page, in the box with 
%the final copies of your thesis.  If you are doing an electronic
%submission, you can ignore the special abstract.

Magnetism is a rich subject touching all aspects of physics. My goal with this dissertation is to explore spin and magnetic dipole moments in \emph{relativistic} mechanics from both a quantum and classical perspective. We emphasize the special case of gyromagnetic ratio $g\!=\!2$ and its relationship to the algebraic spin structure of the wave equations of motion.

In relativistic quantum mechanics, we investigate generalizations of the Dirac equation for arbitrary magnetic dipole moments for fermions. We analyze the homogeneous magnetic field case and the Coulomb problem for hydrogen-like atoms with emphasis on the role of the anomalous magnetic moment (AMM). We explore alternative approaches which combine mass and the magnetic moment. Extensions to include both electromagnetic and quantum chromodynamic (QCD) dipole moments are considered. Classically, we propose a relativistic covariant model of the Stern-Gerlach force via the introduction of a magnetic four-potential. This model modifies the covariant torque equations and unites the Amp{\`e}rian and Gilbertian models for magnetic dipole moments.

We further study (transition) magnetic dipoles in Majorana neutrinos specifically analyzing the relationship between flavor mixing and electromagnetic fields. We demonstrate this explicitly in the 2-flavor model and develop electromagnetic flavor mixing into a dynamical mass basis with an electromagnetic rotation matrix.

An interesting application of these theoretical developments is to study primordial magnetization in the early universe during the hot dense electron-positron plasma epoch. We propose a model of magnetic thermal matter-antimatter plasmas. We analyze the paramagnetic characteristics of electron-positron plasma when exposed to an external primordial field. We determine the magnitude of a small polarization asymmetry is sufficient to generate field strengths in agreement with those measured today in deep intergalactic space.