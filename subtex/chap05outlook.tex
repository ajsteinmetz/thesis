%%%%%%%%%%%%%%%%%%%%%%%%%%%%%%%%%%%%%%%
\chapter{Outlook and key results}
\label{chap:outlook}
%%%%%%%%%%%%%%%%%%%%%%%%%%%%%%%%%%%%%%%

%%%%%%%%%%%%%%%%%%%%%%%%%%%%%%%%%%%%%%%
\subsection*{Chapter 2: Dynamics of charged particles with arbitrary magnetic moment}
\label{sec:chap2}
%%%%%%%%%%%%%%%%%%%%%%%%%%%%%%%%%%%%%%%
We highlight the comparison of magnetic moment dynamics of DP and KGP formulation of relativistic quantum mechanics. The DP equation breaks up the magnetic moment into an underlying spinor structure part inherent to the Dirac equation, and a dedicated anomalous part. In contrast, for the KGP, the entire effect of magnetic moment is contained in a single Pauli term irrespective of the magnetic moment\rq s size. We find that the two models disagree in their predicted energy levels for the homogeneous magnetic field and the Coulomb field.

For the KGP-Landau levels, Eq.\,\eqref{lan24b}, we have a simple dependence on the full magnetic moment $g$-factor and have the correct non-relativistic reduction at lowest order. This simplicity allows, for the KGP equation, the straightforward analysis of physical systems and elegant expressions for their solutions. Thus Dirac's beauty principle favors heavily the KGP considering the Landau levels.

In the case of the Coulomb field there are weak fields differences in both the Lamb shift and fine structure; the contribution to the Lamb shift and fine structure splitting are proportional to: KGP $g^{2}/8-1/2$, Eq.\,\eqref{lamb01} and $g^{2}/8$, Eq.\,\eqref{fs00} respectively, rather than: DP $g/2-1$ Eq.\,\eqref{lamb03}) and $g/2-1/2$, Eq.\,\eqref{fs02}) respectively.

For strong fields, both DP and KGP share the behavior of a shrinking particle/antiparticle gap for the ground state when $|g|>2$, though the expressions differ from each other. The models are extremely different in both their predicted energy values and the ultimate fate of the states as $B$ increases. For the KGP equation, the gap vanishes for in very strong magnetic fields Eq.\,\eqref{Bcrit}, and for DP the fields have to be even stronger.

%%%%%%%%%%%%%%%%%%%%%%%%%%%%%%%%%%%%%%%
\subsection*{Chapter 3: Dynamic neutrino flavor mixing through transition moments}
\label{sec:chap3}
%%%%%%%%%%%%%%%%%%%%%%%%%%%%%%%%%%%%%%%
We have incorporated electromagnetic effects in the Majorana neutrino mixing matrix by introducing an anomalous transition magnetic dipole moment. We have described the formalism for three generations of neutrinos and explicitly explored the two generation case as a toy model. 

In the two generation case, we determined the effect of electric and magnetic fields on flavor rotation in~\req{zrot:2} by introducing an electromagnetic flavor unitary rotation $Z_{kj}^\mathrm{ext}$. We presented remixed mass eigenstates $\widetilde m(E,B)$ in~\req{poly:3} which are the propagating mass-states in a background electromagnetic field $F^{\alpha\beta}_\mathrm{ext}(x^{\mu})$. These EM-mass eigenstates were also further split by spin aligned and anti-aligned states relative to the external field momentum density. There is much left to do to explore further the nascent connection between the spin and the flavor via transition magnetic moments. 

The transition dipole moments are the origin of dynamical flavor mixing. While our focus was on Majorana neutrinos, Dirac-type fermions (neutrinos included) may also have non-zero transition  dipole moments. These could  remix flavor in the presence of strong external background fields. Here  quarks  are of special interest because they are not only electrically charged, but have color charge as well. This means quarks could in principle possess one, or both, EM and color-charge transition dipole moments, leading to dynamical effects in the CKM mixing matrix within hadrons as well as in quark-gluon plasma.

More speculatively, as transition dipoles act as a mechanism to generate mass by virtue of EM energy density $T_\mathrm{ext}^{00}$ as seen in~\req{poly:3}, an analogous consequence of our work could arise in the presence of a dark vector field in the Universe coupled to neutrinos, resulting in off-diagonal masses in flavor. Massless neutrinos could then obtain dynamical non-zero masses in the Universe by virtue of their interactions originating in dark transition moments.

%%%%%%%%%%%%%%%%%%%%%%%%%%%%%%%%%%%%%%%
\subsection*{Chapter 4: Matter-antimatter origin of cosmic magnetism}
\label{sec:chap4}
%%%%%%%%%%%%%%%%%%%%%%%%%%%%%%%%%%%%%%%
We characterized the primordial magnetic properties of the early universe before recombination. We studied the temperature range of $2000\keV$ to $20\keV$ where all of space was filled with a hot dense electron-positron plasma (to the tune of 450 million pairs per baryon) which occurred within the first few minutes after the Big Bang. We note that our chosen period also includes the era of Big Bang Nucleosynthesis.

We found that subject to a primordial magnetic field, the early universe electron-positron plasma has a significant paramagnetic response due to magnetic moment polarization. We considered the interplay of charge chemical potential, baryon asymmetry, anomalous magnetic moment, and magnetic dipole polarization on the nearly homogeneous medium.

We find that electron-positron magnetization rapidly vanishes as the number of pairs depletes as the universe cools. This therefore presents an opportunity for induced currents to facilitate inhomogeneities in the early universe. We also presented a simple model of self-magnetization of the primordial electron-positron plasma which indicates that only a small polarization asymmetry is required to generate significant magnetic flux when the universe was very hot and dense.