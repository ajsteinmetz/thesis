%%%%%%%%%%%%%%%%%%%%%%%%%%%%%%%%%%%%%%%
\chapter{Dynamic neutrino flavor mixing through transition moments}
\label{chap:neutrino}
%%%%%%%%%%%%%%%%%%%%%%%%%%%%%%%%%%%%%%%
We proposed in~\cite{Rafelski:2023zgp} that neutrino flavors are remixed when exposed to strong EM fields travelling as a superposition distinct from the vacuum propagation of free neutrinos. Neutrino mixing is an important topic for studying BSM physics as flavor mixing only occurs in the presence of massive neutrinos allowing for the misalignment between the flavor basis which participates in left-chiral $SU(2)_{L}$ weak interactions and the mass basis which are the propagating neutrino states.

We discuss the neutrino anomalous magnetic moment (AMM) in \rsec{sec:numoment}. We narrow our analysis for Majorana neutrinos which are allowed only transition magnetic moments which couple different flavors electromagnetically, but do not violate CPT symmetry. Transition moments however notably break lepton number conservation. We discuss the standard flavor mixing program in \rsec{sec:nuflavor} and the effective Lagrangian density in \rsec{sec:nuaction} containing both Majorana mass and transition moments. In \rsec{sec:nuem} we discuss the chirality of the relativistic Pauli dipole.

The two-flavor neutrino model is evaluated explicitly in \rsec{sec:nutoy} and the remixed electromagnetic-mass eigenstates are obtained in \rsec{sec:zmixing}. We obtain in \rsec{sec:emmass} an EM-mass basis, distinct from flavor and free-particle mass basis, which mixes flavors as a function of EM fields. The case of two nearly degenerate neutrinos is studied in detail in \rsec{sec:nulimits}. Moreover we show solutions relating to full EM field tensor which result in covariant expressions allowing for both magnetic and electrical fields.

%%%%%%%%%%%%%%%%%%%%%%%%%%%%%%%%%%%%%%%
\section{Electromagnetic characteristics of neutrinos}
\label{sec:numoment}
%%%%%%%%%%%%%%%%%%%%%%%%%%%%%%%%%%%%%%% 
We study the connection between Majorana neutrino transition magnetic dipole moments~\citep{Fujikawa:1980yx,Shrock:1980vy,Shrock:1982sc} and neutrino flavor oscillation. Neutrino electromagnetic (EM) properties have been considered before~\citep{Schechter:1981hw,Giunti:2014ixa,Popov:2019nkr,Dvornikov:2019sfo,Chukhnova:2019oum} including the effect of oscillation in magnetic fields~\citep{Lim:1987tk,Akhmedov:1988uk,Pal:1991pm,Elizalde:2004mw,Akhmedov:2022txm}. The influence of transition magnetic moments on solar neutrinos is expected~\citep{Martinez-Mirave:2023fyb}, but difficult to measure due to the lack of knowledge of solar magnetism near the core.

The case of transition moments has the mathematical characteristics of an off-diagonal mass which is distinct from normal direct dipole moment behavior. EM field effects are also distinct from weak interaction remixing within matter, {\it i.e.\/} the Mikheyev-Smirnov-Wolfenstein effect~\citep{Wolfenstein:1977ue,Mikheyev:1985zog,Mikheev:1986wj,Smirnov:2003da}.

%We then explore the possibility that electromagnetic effects may play a role in CP violation (CPV) in the neutrino sector. We study the effect of strong electromagnetic fields on neutrino CP violation by analysis of the electromagnetic dipole interaction and determining its influence on the Jarlskog invariant $J$ which controls the size of CPV.

%Electromagnetic processes in the neutrino sector may yield measurable effects in two aspects of neutrino physics: (a) neutrino oscillation which is evidence of the non-zero mass eigenstates and (b) the Charge-Parity violation (CPV) in the neutrino sector which occurs due to the presence of at least three generations of neutrinos or additional CP violating interactions.

The size of the neutrino magnetic dipole moment can be constrained as follows: The lower bound is found by higher order standard model interactions with the minimal extension of neutrino mass $m_{\nu}$ included~\citep{Fujikawa:1980yx,Shrock:1980vy,Shrock:1982sc}. The upper bound is derived from reactor, solar and astrophysical experimental observations~\citep{Giunti:2015gga,Canas:2015yoa,Studenikin:2016ykv,AristizabalSierra:2021fuc}. The bounds are expressed in terms of the electron Bohr magneton $\mu_{B}$ as
\begin{align}
    \label{bound:1}
    \frac{e\hbar G_{F}m_{\nu}c^{2}}{8\pi^{2}\sqrt{2}} \sim 10^{-20}\mu_{B}<\mu_{\nu}^\mathrm{eff}<10^{-10}\mu_{B}\,,\qquad\mu_{B}=\frac{e\hbar}{2m_{e}}
\end{align}
where $G_{F}$ is the Fermi constant and $\mu_{\nu}^\mathrm{eff}$ is the effective and characteristic size of the neutrino magnetic moment. In \req{bound:1}, the lower bound was estimated using a characteristic mass of $m_{\nu}\sim0.1~\mathrm{eV}$. From cosmological studies, the sum of neutrino masses is estimated~\citep{Planck:2018vyg} to be $\sum_{i}m_{i}<0.12$~eV; the effective electron (anti)neutrino mass is bounded~\citep{KATRIN:2021uub} by $m_{e}^{\nu}<0.8$~eV.

%%%%%%%%%%%%%%%%%%%%%%%%%%%%%%%%%%%%%%%
\section{Neutrino flavor mixing and electromagnetic fields}
\label{sec:nuflavor}
%%%%%%%%%%%%%%%%%%%%%%%%%%%%%
Oscillation of neutrino flavors observed in experiment is in general interpreted as being due to a difference in neutrino mass and flavor eigenstates. This misalignment between the two representations is described as rotation of the neutrino flavor $N$-vector where $N=3$ is the observed number of generations. The unitary mixing matrix $V_{\ell k}$ allows for the change of basis between mass $(k)$ and flavor $(\ell)$ eigenstates via the transform 
\begin{alignat}{1}
    \label{basis:1} \nu_{\ell}=V_{\ell k}\nu_{k}\,\rightarrow
    \begin{pmatrix}
        \nu_{e}\\
        \nu_{\mu}\\
        \nu_{\tau}
    \end{pmatrix}=
    \begin{pmatrix}
        V_{e1} & V_{e2} & V_{e3}\\
        V_{\mu1} & V_{\mu2} & V_{\mu3}\\
        V_{\tau1} & V_{\tau2} & V_{\tau3}
    \end{pmatrix}
    \begin{pmatrix}
        \nu_{1}\\
        \nu_{2}\\
        \nu_{3}
    \end{pmatrix}\,,
\end{alignat}
where $\nu_{\ell}$ is the neutrino four-spinor written in the flavor basis while in the mass basis we use $\nu_{k}$ with $k\in1,2,3$.

The parameterization of the components of the mixing matrix depends on the Dirac or Majorana-nature of the neutrinos. First we recall the Dirac neutrino mixing matrix $U_{\ell k}$ in the standard parameterization~\citep{Schwartz:2014sze} 
\begin{alignat}{1}
    \label{rotation:1} U_{\ell k} =
    \begin{pmatrix}
         c_{12}c_{13} & s_{12}c_{13} & s_{13}e^{-i\delta}\\
         -s_{12}c_{23} - c_{12}s_{13}s_{23}e^{i\delta} & c_{12}c_{23} - s_{12}s_{13}s_{23}e^{i\delta} & c_{13}s_{23}\\
         s_{12}s_{23} - c_{12}s_{13}c_{23}e^{i\delta}& -c_{12}s_{23} - s_{12}s_{13}c_{23}e^{i\delta} & c_{13}c_{23}
    \end{pmatrix}\,,
\end{alignat}
where $c_{ij} = \mathrm{cos}(\theta_{ij})$ and $s_{ij} = \mathrm{sin}(\theta_{ij})$. In this convention, the three mixing angles $(\theta_{12}, \theta_{13}, \theta_{23})$ are understood to be the Euler angles for generalized rotations and $\delta$ is the CP-violating complex phase. 

For the Majorana case we must allow a greater number of complex phases: Majorana neutrinos allow up to two additional complex phases $\rho$ and $\sigma$ which along with $\delta$ participate in CP-violation. A parameterization is achieved by introducing an additional phase matrix $P_{kk'}$
\begin{alignat}{1}
    \label{phases:1} &V_{\ell k} = U_{\ell k'}P_{k'k}\,,\\
    \label{phases:3} &P_{kk'} = \mathrm{diag}(e^{i\rho},e^{i\sigma},1)\,.
\end{alignat}
The mixing matrix $V_{\ell k}$ defined in \req{phases:1} can then be used to transform the symmetric mass matrix $M_{\ell\ell'}$ from the flavor basis into the diagonal mass basis 
\begin{align}
    \label{diag:0}
    M_{\ell\ell'}=
    \begin{pmatrix}
        m_{ee}^{\nu} & m_{e\mu}^{\nu} & m_{e\tau}^{\nu}\\
        m_{e\mu}^{\nu} & m_{\mu\mu}^{\nu} & m_{\mu\tau}^{\nu}\\
        m_{e\tau}^{\nu} & m_{\mu\tau}^{\nu} & m_{\tau\tau}^{\nu}
    \end{pmatrix}\,,\qquad
    M_{\ell\ell'}^\mathrm{T}=M_{\ell\ell'}\,,\\
    \label{diag:1}
    V_{\ell k}^\mathrm{T}M_{\ell\ell'}V_{\ell'k'} = M_{kk'} = m_{k}\delta_{kk'} = \mathrm{diag}(m_{1},m_{2},m_{3})\,.
\end{align}
We note the Majorana mass matrix is symmetric due to the anticommuting nature of the neutrino fields $\bar\nu\nu=-\nu^\mathrm{T}\bar\nu^\mathrm{T}$ and is in general complex~\citep{Adhikary:2013bma,giunti2007fundamentals} though it will be taken to be fully real in this work. There are many interesting models for mass matrices which were pioneered by~\cite{Fritzsch:1995dj,Fritzsch:1998xs,Fritzsch:1999ee,Xing:2000ik} in the leptonic sector. The masses $m_{k}$ are taken to be real and positive labelling the free propagating states of the three neutrinos.

%%%%%%%%%%%%%%%%%%%%%%%%%%%%%%%%%%%%%%%
\subsection{Effective Majorana neutrino Lagrangian}
\label{sec:nuaction}
%%%%%%%%%%%%%%%%%%%%%%%%%%%%%
Given the mass matrix defined in \req{diag:0}, the Majorana mass term in the Lagrangian can be written in the flavor basis as
\begin{alignat}{1}
    \label{mass:1} -\mathcal{L}_{\mathrm{mass}}^{\mathrm{Maj.}}=\frac{1}{2}\bar\nu_{\ell}M_{\ell\ell'}\nu_{\ell'}=
    -\frac{1}{2}\nu_{L,\ell}^\mathrm{T}C^{\dag}M_{\ell\ell'}\nu_{L,\ell'}+\mathrm{h.c.}\,,
\end{alignat}
where the Majorana fields are written as $\nu=\nu_{L}+C(\bar\nu_{L})^\mathrm{T}$. The field $\nu_{L}$ refers to left-handed Weyl four-component spinors. Charged conjugated fields are written as $\nu^{c}=C(\bar\nu)^\mathrm{T}$. The charge conjugation operator $C$ is defined in the usual way in~\cite{Itzykson:1980rh}; p.692.

Given these conventions, we can extend our consideration to include the electromagnetic interaction of neutrinos which is possible if neutrinos are equipped with a magnetic moment matrix $\mu_{\ell\ell'}$. We allow for a fixed \emph{external} electromagnetic field tensor $F^{\alpha\beta}_\mathrm{ext}(x^{\mu})$ which imparts a force on the neutrino fields. We emphasize that $F^{\alpha\beta}_\mathrm{ext}$ is not dynamical in our formulation and consists of real functions over four-position and not field operators.

We generalize the AMM Pauli Lagrangian in \req{lamm:1} to account for the Majorana fields in the flavor basis as
\begin{align}
\label{moment:1}
    -\mathcal{L}_{\mathrm{AMM}}^\mathrm{Maj.}=\frac{1}{2}\bar\nu_{\ell}\left(\mu_{\ell\ell'}\frac{1}{2}\sigma_{\alpha\beta}F^{\alpha\beta}_\mathrm{ext}\right)\nu_{\ell'}=
    -\frac{1}{2}\nu_{L,\ell}^\mathrm{T}C^{\dag}\left(\mu_{\ell\ell'}\frac{1}{2}\sigma_{\alpha\beta}F^{\alpha\beta}_\mathrm{ext}\right)\nu_{L,\ell'}+\mathrm{h.c.}
\end{align}
The operator $\sigma_{\alpha\beta}$ is the $4\times 4$ spin tensor defined in \req{sigma:1}. We would like to point out some interesting features of the Pauli term most notably that the spin tensor itself is not Hermitian with
\begin{align}
\label{notherm:1}
\sigma_{\alpha\beta}^{\dag} = \gamma_{0}\sigma_{\alpha\beta}\gamma_{0}\,.
\end{align}
However, the conjugate of the Lagrangian term in \req{moment:1}
\begin{align}
\left(\nu^{\dag}\gamma_{0}\sigma_{\alpha\beta}F^{\alpha\beta}_\mathrm{ext}\nu\right)^{\dag} = \nu^{\dag}\sigma_{\alpha\beta}^{\dag}F^{\alpha\beta}_\mathrm{ext}\gamma_{0}\nu = \nu^{\dag}\gamma_{0}\sigma_{\alpha\beta}F^{\alpha\beta}_\mathrm{ext}\nu\,,
\end{align}
is Hermitian. More about the spin tensor's properties will be elaborated on in \rsec{sec:numoment}.

The Majorana magnetic moment matrix acts in flavor space. It satisfies the following constraints~\citep{Giunti:2014ixa} for CPT symmetry reasons and the anticommuting nature of fermions
\begin{alignat}{1}
\label{props:1}
\mu_{\ell\ell'}^{\dag}=\mu_{\ell\ell'}\,,\qquad
\mu_{\ell\ell'}^\mathrm{T}=-\mu_{\ell\ell'}\,,
\end{alignat}
{\it i.e.\/} the AMM matrix $\mu_{\ell\ell'}$ is Hermitian and fully anti-symmetric. This requires that the transition magnetic moment elements are purely imaginary while all diagonal AMM matrix elements vanish
\begin{align}
\label{mu:1}
\mu_{\ell\ell'}=
\begin{pmatrix}
\mu_{ee} & \mu_{e\mu} & \mu_{e\tau} \\
\mu_{\mu e} & \mu_{\mu\mu} & \mu_{\mu\tau} \\
\mu_{\tau e} & \mu_{\tau\mu} & \mu_{\tau\tau}
\end{pmatrix}\xrightarrow{\mathrm{Majorana}}
\mu_{\ell\ell'}=
\begin{pmatrix}
0 & i\mu_{e\mu} & -i\mu_{e\tau} \\
-i\mu_{e\mu} & 0 & i\mu_{\mu\tau} \\
i\mu_{e\tau} & -i\mu_{\mu\tau} & 0
\end{pmatrix}\,.
\end{align}

We can combine the mass term in~\req{mass:1} and AMM contribution in~\req{moment:1} into a single effective Lagrangian
\begin{align}
\label{massmom:1}
\mathcal{L}_\mathrm{eff}^\mathrm{Maj.} &= \mathcal{L}_\mathrm{kinetic}^\mathrm{Maj.} + \mathcal{L}_\mathrm{mass}^\mathrm{Maj.} + \mathcal{L}_\mathrm{AMM}^\mathrm{Maj.}\,,\\
\label{massmom:2}
\mathcal{L}_\mathrm{eff}^\mathrm{Maj.} &= \mathcal{L}_\mathrm{kinetic}^\mathrm{Maj.} - \frac{1}{2}\bar\nu_{\ell}\left(M_{\ell\ell'}+\mu_{\ell\ell'}\frac{1}{2}\sigma_{\alpha\beta}F^{\alpha\beta}_\mathrm{ext}\right)\nu_{\ell'}\;.
\end{align}
\req{massmom:2} is our working Lagrangian. For later convenience we define the generalized mass-dipole matrix $\mathcal{M}_{\ell\ell'}$ present in \req{massmom:2} as
\begin{align}
\label{massmom:3}
\mathcal{M}_{\ell\ell'}(\bb{E},\bb{B})\equiv M_{\ell\ell'}+\mu_{\ell\ell'}\frac{1}{2}\sigma_{\alpha\beta}F^{\alpha\beta}_\mathrm{ext}\,,\qquad \mathcal{M}_{\ell\ell'}^{\dag}=\gamma_{0}\mathcal{M}_{\ell\ell'}\gamma_{0}\,.
\end{align}
As neutrinos must propagate as energy eigenstates, our objective is to find the eigenvalues of \req{massmom:2} rather than \req{diag:1}. As the mass eigenvalues are modified by the presence the EM interactions $m\rightarrow\widetilde m(\bb{E},\bb{B})$ so will the mixing matrix, leading to modifications of \req{phases:1}. These electromagnetic components then facilitate time-dependant oscillation among the free-particle mass eigenstates~\citep{Giunti:2014ixa}.

Additionally we may consider matter effects via the weak interaction. Electron (anti)neutrinos passing through matter preferentially interact via weak charge-current (via the $W^{\pm}$ boson) with electrons which make up the bulk of charged leptons in most matter. The neutral-current (via the $Z_{0}$ boson) however affects all flavors and couples to the neutrons within the medium as the electron and proton contributions cancel in charge neutral matter. This can be represented, MSW effect aside, as the weak charge-current $V_{CC}$ and neutral-current $V_{NC}$ effective potentials~\citep{Pal:1991pm,greiner2009gauge} which contribute to the action as
\begin{align}
\label{matter:1}
\mathcal{L}_\mathrm{matter}^\mathrm{Maj.} &= \bar\nu_{\ell}(\gamma_{0}V_{\ell\ell'})\nu_{\ell'}\,,\qquad
V_{\ell\ell'} = 
\begin{pmatrix}
V_{CC}+V_{NC} & 0 & 0\\
0 & V_{NC} & 0\\
0 & 0 & V_{NC}
\end{pmatrix}\,,\\
V_{CC} &= \sqrt{2}G_{F}\hbar^{2}c^{2}n_{e}\,,\qquad V_{NC} = -\frac{1}{2}\sqrt{2}G_{F}\hbar^{2}c^{2}n_{n}\,.
\end{align}
The coefficient $G_{F}$ is the Fermi constant, $n_{e}$ is the number density of electron matter and $n_{n}$ is the number density of neutrons within the medium. We note that $V_{\ell\ell'}\gamma_{0}$ behaves like the zeroth component of a vector-potential. As written, \req{matter:1} is approximately true for non-relativistic matter.

%%%%%%%%%%%%%%%%%%%%%%%%%%%%%%%%%%%%%%%
\subsection{Chiral properties of the relativistic Pauli dipole}
\label{sec:nuem}
%%%%%%%%%%%%%%%%%%%%%%%%%%%%%
While the Pauli dipole was introduced and discussed in \rsec{sec:dp}, we will further elaborate on details directly relevant to neutrinos. The electromagnetic dipole behavior of the neutrino depends on mathematical properties of the tensor product $\sigma_{\alpha\beta}F^{\alpha\beta}_\mathrm{ext}$. We prefer to work in the Weyl (chiral) spinor representation where the EM contribution is diagonal in spin space. Therefore we evaluate the product $\sigma_{\alpha\beta}F^{\alpha\beta}_\mathrm{ext}$ in the Weyl representation following~\cite{Feynman:1958ty} yielding
\begin{align}
\label{chiral:1}
-\frac{1}{2}\sigma_{\alpha\beta}F^{\alpha\beta}_\mathrm{ext}=
\begin{pmatrix}
\bb{\sigma}\cdot(\bb{B}+i\bb{E}/c) & 0\\
0 & \bb{\sigma}\cdot(\bb{B}-i\bb{E}/c)
\end{pmatrix}\equiv
\begin{pmatrix}
\bb{\sigma}\cdot\bb{f}_{+} & 0 \\
0 & \bb{\sigma}\cdot\bb{f}_{-}
\end{pmatrix}\,,
\end{align}
where we introduced the complex electromagnetic field form $\bb{f}_{\pm}=\bb{B}\pm i\bb{E}/c$ showing sensitivity to both magnetic and electric fields. The eigenvalues of \req{chiral:1} were also discussed in \rsec{sec:ikgp}. As this expression is diagonal in the Weyl representation, it does not exchange handedness when acting upon a state. This is explicitly understood by the fact that \req{chiral:1} commutes with $\gamma^{5}$. Since left and right-handed neutrinos are not remixed by magnetic moments, sterile right-handed neutrinos do not need to be introduced. We can also see explicitly in \req{chiral:1} its non-Hermitian character, see \req{massmom:2}, of the EM spin-field coupling. Specifically this is mirrored in the complex field's $\bb{f}_{\pm}$ relation to its complex conjugate $(\bb{f}_{\pm})^{*}=\bb{f}_{\mp}$. The complex EM fields have a Hermitian $(\bb{B})$ and anti-Hermitian $(i\bb{E})$ part.

Taking the product of $\bb{f}_{\pm}$ with its complex conjugate we find
\begin{align}
\label{cross:1}
\frac{1}{2}\left(\bb{\sigma}\cdot\bb{f}_{\pm}\right)\left(\bb{\sigma}\cdot\bb{f}_{\mp}\right)=T_\mathrm{ext}^{00}\mp \sigma_{i}T_\mathrm{ext}^{0i}\,,
\end{align}
where we recognize the stress-energy tensor $T_\mathrm{ext}^{\alpha\beta}$ component $T_\mathrm{ext}^{00}$ for field energy density and $T_\mathrm{ext}^{0i}$ momentum density respectively
\begin{align}
T_\mathrm{ext}^{00}=\frac{1}{2}\left(B^{2}+E^{2}/c^{2}\right)\,,\qquad
T_\mathrm{ext}^{0i}=\frac{1}{c}\varepsilon_{ijk}E_{j}B_{k}\,.
\end{align}
As we will see in \rsec{sec:nutoy}, \req{cross:1} will appear in the EM-mass eigenvalues of our effective Lagrangian \req{massmom:1}. Using the identity in \req{chiral:1} and \req{cross:1} we also find the interesting relationship
\begin{align}
\label{cross:2}
\frac{1}{2}\left(\frac{1}{2}\sigma_{\alpha\beta}F^{\alpha\beta}_\mathrm{ext}\right)\left(\frac{1}{2}\sigma_{\alpha\beta}F^{\alpha\beta}_\mathrm{ext}\right)^{\dag}=
\gamma_{0}\left(T_\mathrm{ext}^{00}\gamma_{0}+T_\mathrm{ext}^{0i}\gamma_{i}\right)\,.
\end{align}
Now that we have elaborated on the relevant EM field identities, we turn back to the magnetic dipole and flavor rotation problem.

%%%%%%%%%%%%%%%%%%%%%%%%%%%%%%%%%%%%%%%
\section{Electromagnetic-flavor mixing for two generations}
\label{sec:nutoy}
%%%%%%%%%%%%%%%%%%%%%%%%%%%%%%%%%%%%%%%
Considering experimental data on neutrino oscillations, it is understood that either the two lighter (normal hierarchy) or the two heavier (inverted hierarchy) neutrino states are close together in mass. If the electromagnetic properties of the neutrino do indeed lead to flavor mixing effects, then it is likely the closer pair of neutrino mass states that are most sensitive to the phenomenon we explore. In the spirit of~\cite{Bethe:1986ej}, we therefore explore the $N=2$ two generation $(\nu_{e},\nu_{\mu})$ toy model.

Following the properties established in \req{props:1} and \req{massmom:3} we write down the two generation mass and dipole matrices as
\begin{alignat}{1}
\label{mix:1} M_{\ell\ell'}= 
\begin{pmatrix}
m_{e}^{\nu} & {\delta m}\\
{\delta m} & m_{\mu}^{\nu}
\end{pmatrix}\,,\qquad
\mu_{\ell\ell'} = 
\begin{pmatrix}
0 & i\delta\mu\\
-i\delta\mu & 0
\end{pmatrix}\,.
\end{alignat}
The AMM coupling $\delta\mu$ is taken to be real with a pure imaginary coefficient. While the mass elements $(m_{e}^{\nu},m_{\mu}^{\nu},{\delta m})$ are generally complex, we choose in our toy model for them to be fully real
\begin{align}
\label{choice:1}
m_{e}^{\nu}=(m_{e}^{\nu})^{*}\,,\qquad
m_{\mu}^{\nu}=(m_{\mu}^{\nu})^{*}\,,\qquad
\delta m=\delta m^{*}\,,
\end{align}
making the mass matrix $M_{\ell\ell'}$ Hermitian. This allows us to more easily evaluate and emphasize the EM contributions to mixing rather than complications arising from the mass matrix.

Using \req{mix:1} and \req{choice:1}, we write the mass-dipole matrix in \req{massmom:3} in terms of $2\times2$ flavor components as
\begin{align}
\label{mix:2}
\mathcal{M}_{\ell\ell'} = 
\begin{pmatrix}
m_{e}^{\nu} & {\delta m}+i\delta\mu\sigma_{\alpha\beta}F^{\alpha\beta}_\mathrm{ext}/2\\
{\delta m}-i\delta\mu\sigma_{\alpha\beta}F^{\alpha\beta}_\mathrm{ext}/2 & m_{\mu}^{\nu}
\end{pmatrix}\,,\qquad
\mathcal{M}_{\ell\ell'}^{\dag}=\gamma_{0}\mathcal{M}_{\ell\ell'}\gamma_{0}\,.
\end{align}
As noted before, this matrix is not Hermitian due to the inclusion of the spin tensor, therefore it is not guaranteed to satisfy an algebraic eigenvalue equation in its present form which is a requirement for well behaved masses.

This can be remedied by recalling that any arbitrary complex matrix can be diagonalized into its real eigenvalues $\lambda_{j}$ by the biunitary transform
\begin{align}
\label{biunitary:1}
W_{\ell j}^{\dag}\mathcal{M}_{\ell\ell'}Y_{\ell'j'}=\lambda_{j}\delta_{jj'}\,,
\end{align}
where $Y_{\ell j}$ and $W_{\ell j}$ are both unitary matrices. Taking the complex conjugate of \req{biunitary:1}, we arrive at
\begin{align}
\label{biunitary:2}
(W_{\ell j}^{\dag}\mathcal{M}_{\ell\ell'}Y_{\ell'j'})^{\dag} = 
Y_{\ell j'}^{\dag}\gamma_{0}\mathcal{M}_{\ell\ell'}\gamma_{0}W_{\ell' j}=\lambda_{j}\delta_{jj'}\,,\\
Y_{\ell j}=\gamma_{0}W_{\ell j}\rightarrow
W_{\ell j}^{\dag}\mathcal{M}_{\ell\ell'}\gamma_{0}W_{\ell'j'}=\lambda_{j}\delta_{jj'}\,. 
\end{align}
As $Y_{\ell j}$ and $W_{\ell j}$ are related by a factor of $\gamma_{0}$ based on the conjugation properties of \req{mix:2}, this lets us eliminate $Y_{\ell j}$ and diagonalize using a single unitary matrix $W_{\ell j}$. The related matrix $\mathcal{M}_{\ell\ell'}\gamma_{0}$ is Hermitian
\begin{align}
\label{herm:1}
(\mathcal{M}_{\ell\ell'}\gamma_{0})^{\dag} = \mathcal{M}_{\ell\ell'}\gamma_{0}\,,
\end{align}
and also equivalent to the root of the Hermitian product of \req{mix:2}
\begin{align}
(\mathcal{M}\mathcal{M}^{\dag})_{\ell\ell'} = \left((\mathcal{M}\gamma_{0})(\mathcal{M}\gamma_{0})\right)_{\ell\ell'}\,.
\end{align}
Therefore a suitable unitary transformation $W_{\ell j}$ rotates flavor $\ell$-states into magnetized mass $j$-states. The eigenvalues $\lambda_{j}^{2}$ of $(\mathcal{M}\mathcal{M}^{\dag})_{\ell\ell'}$ are the squares of both signs of the eigenvalues of $\mathcal{M}_{\ell\ell'}\gamma_{0}$. We write this property (with flavor indices suppressed) as
\begin{align}
W^{\dag}(\mathcal{M}\mathcal{M}^{\dag})W &= W^{\dag}(\mathcal{M}\gamma_{0})WW^{\dag}(\mathcal{M}\gamma_{0})W = \mathrm{diag}(\lambda_{1}^{2},\lambda_{2}^{2})\,.
\end{align}
We associate $\lambda_{j}=\widetilde m_{j}(\bb{E},\bb{B})$ with $j\in1,2$ as the effective EM-mass states which are field dependant in this basis. 

%%%%%%%%%%%%%%%%%%%%%%%%%%%%%%%%%%%%%%%
\subsection{Separating electromagnetic-mass mixing into two rotations}
\label{sec:zmixing}
%%%%%%%%%%%%%%%%%%%%%%%%%%%%%%%%%%%%%%%
The matrix $W_{\ell j}$ mixes flavor states into a new basis distinct from the free-particle case however this rotation must smoothly connect with the free-particle case in the limit that the electromagnetic fields go to zero. We proceed to evaluate $W_{\ell j}$ breaking the rotation into two separate unitary transformations:
\begin{itemize}[nosep]
    \item [(a)] Rotation matrix $V_{\ell k}^{\dag}(\ell\rightarrow k)$ converting from flavor to free-particle mass
    \item [(b)] Rotation matrix $Z_{kj}^{\mathrm{ext}\dag}(k\rightarrow j)$ converting from free-particle mass to EM-mass
\end{itemize}
Guided by \req{basis:1} we write
\begin{align}
\label{zrot:1}
\nu_{j} = W^{\dag}_{\ell j}\nu_{\ell} = Z_{kj}^{\mathrm{ext}\dag}V_{\ell k}^{\dag}\nu_{\ell}\,.
\end{align}
In the limit that the EM fields go to zero, the electromagnetic rotation becomes unity $Z_{kj}^\mathrm{ext}\rightarrow\delta_{kj}$ thereby ensuring the EM-mass basis and free-particle mass basis become equivalent. The rotation $Z_{kj}^\mathrm{ext}$ can then be interpreted as the external field forced rotation. While our argument above is done explicitly for the two generation case, it can be generalized to accommodate three generations of neutrinos as well.

According to \req{diag:1}, the mass matrix in \req{mix:1} can be diagonalized in the two generation case by a one parameter unitary mixing matrix $V_{\ell k}$ given by
\begin{align}
\label{rot:1}
V_{\ell k}(\theta)=
\begin{pmatrix}
\cos\theta & \sin\theta\\
-\sin\theta & \cos\theta
\end{pmatrix}\,.
\end{align}
For a real Hermitian $2\times 2$ mass matrix, the rotation matrix $V_{\ell k}$ is real and only depends on the angle $\theta$. The explicit form of the EM-field related rotation $Z_{kj}^\mathrm{ext}$ introduced in \req{zrot:1} is
\begin{align}
\label{zrot:2}
Z_{kj}^\mathrm{ext}(\omega,\phi)=
\begin{pmatrix}
\cos\omega & e^{i\phi}\sin\omega\\
-e^{-i\phi}\sin\omega & \cos\omega
\end{pmatrix}\,,\qquad
W_{\ell j}(\theta,\omega,\phi)=V_{\ell k}(\theta)Z_{kj}^\mathrm{ext}(\omega,\phi)\,,
\end{align}
where $Z_{kj}^\mathrm{ext}$ depends on the real angle $\omega$ and complex phase $\phi$. The full rotation $W_{\ell j}$ therefore depends on three parameters when broken into free-particle rotation and EM rotation.

The eigenvalues of the original Hermitian mass matrix in \req{mix:1} are given by
\begin{align}
\label{massroot:1}
m_{1,2}=\frac{1}{2}\left(m_{e}^{\nu}+m_{\mu}^{\nu}\mp\sqrt{|\Delta m_{0}|^{2}+4\delta m^{2}}\right)\,,\qquad
|\Delta m_{0}|=|m_{\mu}^{\nu}-m_{\mu}^{e}|\,.
\end{align}
We assign $m_{1}$ to the lower mass $(-)$ root and $m_{2}$ with the larger mass $(+)$ additive root. The rotation angle $\theta$ in \req{rot:1} is then given by
\begin{align}
\label{massroot:2}
\sin2\theta=\sqrt{\frac{4\delta m^{2}}{|\Delta m_{0}|^{2}+4\delta m^{2}}}\,,\qquad
\cos2\theta=\sqrt{\frac{|\Delta m_{0}|^{2}}{|\Delta m_{0}|^{2}+4\delta m^{2}}}\,.
\end{align}

In our toy model, the off-diagonal imaginary transition magnetic moment  $\mu_{\ell\ell'}$ commutes with the real valued mixing matrix $V_{\ell k}$ and the following relations hold
\begin{align}
\label{commuting:1}
V_{\ell k}^{\dag}\mu_{\ell\ell'}V_{\ell' k'}=(V^{\dag}V)_{k\ell'}\mu_{\ell'k'}=\mu_{kk'}=
\begin{pmatrix}
0 & i\delta\mu\\
-i\delta\mu & 0
\end{pmatrix}\,.
\end{align}
We see that the Majorana transition dipoles in our model are off-diagonal in both flavor and mass basis. Therefore the real parameter unitary matrix in \req{commuting:1} cannot rotate a pure imaginary matrix at least in the two generation case. We apply the rotation in \req{rot:1} to \req{herm:1} yielding
\begin{align}
\label{herm:2}
V_{\ell k}^{\dag}(\mathcal{M}_{\ell\ell'}\gamma_{0})V_{\ell' k'} &= 
V_{\ell k}^{\dag}M_{\ell\ell'}\gamma_{0}V_{\ell' k'} +
V_{\ell k}^{\dag}(\mu_{\ell\ell'}\sigma_{\alpha\beta}\gamma_{0}F^{\alpha\beta}_\mathrm{ext}/2)V_{\ell' k'}\,,\\
\label{herm:3}
V_{\ell k}^{\dag}(\mathcal{M}_{\ell\ell'}\gamma_{0})V_{\ell' k'} &= 
\begin{pmatrix}
m_{1}\gamma_{0} & i\delta\mu\sigma_{\alpha\beta}\gamma_{0}F^{\alpha\beta}_\mathrm{ext}/2\\
-i\delta\mu\sigma_{\alpha\beta}\gamma_{0}F^{\alpha\beta}_\mathrm{ext}/2 & m_{2}\gamma_{0}
\end{pmatrix}\equiv
\begin{pmatrix}
\mathcal{A} & i\mathcal{C}\\
-i\mathcal{C} & \mathcal{B}
\end{pmatrix}\,,
\end{align}
where we have defined implicitly the Hermitian elements $(\mathcal{A},\mathcal{B},\mathcal{C})$.  Applying now both rotations 
to \req{herm:1} yields
\begin{align}
    \label{herm:4}
    W_{\ell j}^{\dag}(\mathcal{M}_{\ell\ell'}\gamma_{0})W_{\ell' j'} &= 
    Z^{\mathrm{ext}\dag}
    \begin{pmatrix}
        \mathcal{A} & i\mathcal{C}\\
        -i\mathcal{C} & \mathcal{B}
    \end{pmatrix}Z^\mathrm{ext}=\lambda_{j}\delta_{jj'}\,.
\end{align}
\req{herm:4} is therefore the working matrix equation which needs to be solved to identify the EM rotation parameters. As discussed before, this means that the rotation angle $\omega$ and phase $\phi$ are in general functions of electromagnetic fields.

%%%%%%%%%%%%%%%%%%%%%%%%%%%%%%%%%%%%%%%
\subsection{Effective electromagnetic-mass eigenvalues}
\label{sec:emmass}
%%%%%%%%%%%%%%%%%%%%%%%%%%%%%%%%%%%%%%%
We will now solve for the rotation parameters necessary to define the EM-mass basis which acts as a distinct propagating basis for neutrinos in external fields. Considering that the $j$-columns vectors $\bb{v}^{(j)}$ of $Z_{kj}^\mathrm{ext}$ as eigenvectors for each eigenvalue $\lambda_{j}$
\begin{align}
    \label{herm:5}
    Z_{kj}^\mathrm{ext}=v_{k}^{(j)}=
    \begin{pmatrix}
        \bb{v}^{1} & \bb{v}^{2}
    \end{pmatrix}\,,
\end{align}
\req{herm:4} has the meaning of an eigenvalue equation
\begin{align}
    \label{herm:6}
    \begin{pmatrix}
        \mathcal{A} & i\mathcal{C}\\
        -i\mathcal{C} & \mathcal{B}
    \end{pmatrix}Z^\mathrm{ext}=
    Z^\mathrm{ext}\begin{pmatrix}
        \lambda_{1} & 0\\
        0 & \lambda_{2}
    \end{pmatrix}\rightarrow
    \begin{pmatrix}
        \mathcal{A} & i\mathcal{C}\\
        -i\mathcal{C} & \mathcal{B}
    \end{pmatrix}\bb{v}^{(j)}=\lambda_{j}\bb{v}^{(j)}\,.
\end{align}
Given the eigenvalue equation defined in \req{herm:6}, the effective EM-masses are then solutions to the characteristic polynomial
\begin{align}
\label{poly:1}
(\mathcal{A}-\lambda_{j}\gamma_{0})(\mathcal{B}-\lambda_{j}\gamma_{0})-\mathcal{C}^{2}=0\,,
\end{align}
which we obtained by taking the determinant of \req{herm:6} over flavor but not spin space. It is useful to define the following identities for the off-diagonal element
\begin{align}
\label{poly:1a}
\mathcal{C}^{2} = 
\delta\mu^{2}\left(\frac{1}{2}\sigma_{\alpha\beta}F^{\alpha\beta}_\mathrm{ext}\right)\left(\frac{1}{2}\sigma_{\alpha\beta}F^{\alpha\beta}_\mathrm{ext}\right)^{\dag}=
2\delta\mu^{2}\gamma_{0}\left(T_\mathrm{ext}^{00}\gamma_{0}+T_\mathrm{ext}^{0i}\gamma_{i}\right)\,,
\end{align}
and for the diagonal elements
\begin{align}
(\mathcal{B}-\mathcal{A})^{2} = |m_{2}-m_{1}|^{2} = |\Delta m|^{2}\,,\qquad (\mathcal{A}+\mathcal{B})\gamma_{0} = m_{1} + m_{2}\,.
\end{align}
\req{poly:1a} was obtained using the expression in \req{cross:2}. Because of the spinor behavior of each element, the eigenvalues are obtained with $\gamma_{0}$ coefficients. \req{poly:1} therefore has the roots $\lambda_{1,2} = \widetilde m_{1,2}(\bb{E},\bb{B})$
\begin{align}
\label{poly:2}
\widetilde m_{1,2}(\bb{E},\bb{B})\! =\! \frac{1}{2}\left(m_{1}\!+\!m_{2}\!\mp\!\sqrt{|\Delta m|^{2}\!+\!8\delta\mu^{2}\gamma_{0}\left(T_\mathrm{ext}^{00}\gamma_{0}+T_\mathrm{ext}^{0i}\gamma_{i}\right)}\right)\!,\\
\label{poly:3}
\boxed{\widetilde m_{1,2}(\bb{E},\bb{B})\! =\! \frac{m_{1}\!+\!m_{2}}{2}\!\mp\!\frac{1}{2}\sqrt{|\Delta m|^{2}\!+\!8\delta\mu^{2}\gamma_{0}\left(\gamma_{0}\frac{1}{2}\left(B^{2}\!+\!\frac{E^{2}}{c^{2}}\right)\!+\!\bb{\gamma}\!\cdot\!(\frac{\bb{E}}{c}\times\bb{B})\right)}}
\end{align}
The EM-mass eigenstates $\widetilde m(\bb{E},\bb{B})$ depends on the energy density $T_\mathrm{ext}^{00}$ of the EM field and the spin projection along the EM momentum density $T_\mathrm{ext}^{0i}$. However the coefficient $\delta\mu^{2}$ is presumed to be very small, therefore the EM contribution only manifests in strong EM fields or where the free-particle case has very nearly or exactly degenerate masses, $\Delta m\to 0$. When the the electromagnetic fields go to zero, the EM-masses in \req{poly:3} reduce as expected to the free-particle result.

The complex phase in \req{zrot:2} has the value $\phi=\pi(n-1/2)$ with $n\in0,\pm1,\pm2...$ making the complex exponential in \req{zrot:2} pure imaginary. Curiously, the phase is not field dependant, but tied to the fact that the Majorana moments are pure imaginary quantities. Complex phases in mixing matrices are generally associated with CP violation such as the Dirac phase $\delta$ in \req{rotation:1} which suggests that CP violation in the neutrino sector can be induced in the presence of external EM fields. Some implications of CP violation from transition moments are discussed in~\cite{Nieves:1981zt}. Analysis of the three generation case is required to show this explicitly, but we postulate that the constant valued complex phases would be replaced with field dependant quantities $\delta\to\delta(\bb{E},\bb{B})$.

We note that the solution in \req{poly:3} actually contain four distinct EM-mass eigenstates $\widetilde m_{j}^{s}(\bb{E},\bb{B})$ with the lower $(j=1)$ and upper $(j=2)$ masses and the additional spin splitting from the alignment $(s=+1)$ or anti-alignment $(s=-1)$ of the neutrino spin with the momentum density of the external EM field. Spin splitting vanishes for the pure electric or magnetic field cases. For good spin eigenstates $s\in\pm1$, we can rewrite \req{poly:1a} with EM fields explicitly as
\begin{align}
\label{spinsplit:1}
\mathcal{C}^{2}_{s}(\bb{E},\bb{B})=2\delta\mu^{2}\left(\frac{1}{2}(B^{2}+E^{2}/c^{2})+s|\bb{E}/c\times\bb{B}|\right)\,.
\end{align}
The above expression within the square is positive definite, therefore \req{spinsplit:1} is always real. Spin splitting requires that we consider separate rotations for each spin state as the rotation angle $\omega_{s}$ depends on the spin quantum number
\begin{align}
\label{zrot:3}
\sin2\omega_{s}=\sqrt{\frac{4\mathcal{C}_{s}^{2}}{|\Delta m|^{2}+4\mathcal{C}_{s}^{2}}}\,,\qquad
\cos2\omega_{s}=\sqrt{\frac{|\Delta m|^{2}}{|\Delta m|^{2}+4\mathcal{C}_{s}^{2}}}\,.
\end{align}
The expressions in \req{zrot:3} are mathematically similar to that of the free-particle case written in \req{massroot:2} in the two flavor generation model with the off-diagonal mass being replaced with the EM dependant quantity $\mathcal{C}_{s}$.

%%%%%%%%%%%%%%%%%%%%%%%%%%%%%%%%%%%%%%%
\section{Strong field (degenerate mass) and weak field limits}
\label{sec:nulimits}
%%%%%%%%%%%%%%%%%%%%%%%%%%%%%%%%%%%%%%%
The rotation angles in \req{zrot:3} reveal two distinct limits where EM-masses are dominated by either:
\begin{itemize}[nosep]
    \item[(a)] Intrinsic mass splitting $\mathcal{C}_{s}\ll|\Delta m|^{2}$ with $\omega_{s}\rightarrow0$
    \item[(b)] EM mass splitting $\mathcal{C}_{s}\gg|\Delta m|^{2}$ with $\omega_{s}\rightarrow\pi/4$
\end{itemize}
For the first case where the masses are not degenerate or the fields are weak, we obtain the expansion
\begin{align}
\label{series:1}
\lim_{\mathcal{C}_{s}\ll|\Delta m|^{2}}\widetilde m_{1,2}^{s}(E,B)=\frac{1}{2}\left(m_{1}+m_{2}\mp|\Delta m|\left(1+\frac{2\mathcal{C}_{s}^{2}}{|\Delta m|^{2}}+\ldots\right)\right)\,,
\end{align}
which as stated before reduces to the free-particle case at lowest order.

In the opposite limit, where the masses are very nearly degenerate or fields are strong, the EM-mass eigenvalues in \req{poly:3} can be approximated by the series
\begin{align}
\label{series:2}
\lim_{\mathcal{C}_{s}\gg|\Delta m|^{2}}\widetilde m_{1,2}^{s}(E,B)=\frac{1}{2}\left(m_{1}+m_{2}\mp2\mathcal{C}_{s}\left(1+\frac{|\Delta m|^{2}}{8\mathcal{C}_{s}^{2}}+\ldots\right)\right)
\end{align}
For fully degenerate free-particle masses $m_{1}=m_{2}$, this reduces to
\begin{align}
\label{series:2a}
\lim_{|\Delta m|^{2}\to0}\widetilde m_{1,2}^{s}(E,B)=m_{1}\mp\mathcal{C}_{s}\,.
\end{align}
\req{series:2a} indicates that for degenerate free-particle masses, the effective splitting $|\Delta m_\mathrm{EM}|\equiv\mathcal{C}_{s}$ between masses arises purely from the electromagnetic interaction of the neutrinos. We return to this interesting insight in our final comments.

Because of the bounds in \req{bound:1} on the effective neutrino magnetic moment, we can estimate the field strength required for an external magnetic field to generate an electromagnetic mass splitting of $|\Delta m_\mathrm{EM}|=10^{-3}$~eV which is a reasonable comparison to intrinsic splitting based on the experimental limits on neutrino masses. Using the upper limit for the neutrino effective moment of $\mu_{\nu}^\mathrm{eff}\sim10^{-10}\mu_{B}$ we obtain
\begin{align}
\label{estimate:1}
\left.\frac{\mathcal{C}_{s}}{\mu_{\nu}^\mathrm{eff}}\right\rvert_{\vec{E}=0}=\frac{10^{-3}\,\mathrm{eV}}{10^{-10}\mu_{B}}\approx1.7\times10^{11}\,\mathrm{T}\,.
\end{align}
This is near the upper bound of the magnetic field strength of magnetars~\citep{Kaspi:2017fwg} which are of the order $10^{11}$~Tesla. In this situation, the EM contribution to the mass splitting rivals the estimated inherent splitting~\citep{ParticleDataGroup:2022pth} of the two closer in mass neutrinos. Primordial magnetic fields~\citep{Grasso:2000wj} in the Early Universe may also present an environment for significant EM neutrino flavor mixing as both the external field strength and density of neutrinos would be very large~\citep{Rafelski:2023emw}. The magnetic properties of neutrinos may also have contributed alongside the charged leptons in magnetization in the Early Universe~\citep{Steinmetz:2023nsc} prior to recombination. 

While the above estimate was done with astrophysical systems in mind, we note that strong electrical fields should also produce EM-mass splitting. Therefore environments near to high $Z$-nuclei is also of interest~\citep{Bouchiat:1974kt,Bouchiat:1997mj,Safronova:2017xyt} as weak interactions violate parity. Should neutrinos have abnormally large transition magnetic dipole moments, then they should exhibit mass splitting from the neutrino's electromagnetic dipole interaction which may compete with the intrinsic mass differences of the free-particles.