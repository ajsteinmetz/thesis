%%%%%%%%%%%%%%%%%%%%%%%%%%%%%%%%%%%%%%%
\chapter*{Publications and author contributions}
\label{sec:pubs}
\addcontentsline{toc}{chapter}{PUBLICATIONS AND AUTHOR CONTRIBUTIONS}
%%%%%%%%%%%%%%%%%%%%%%%%%%%%%%%%%%%%%%%
In the course of satisfying the University of Arizona Department of Physics's requirements for a Ph.D. doctoral dissertation, I prepared the following publications which are reprinted in full in the appendices. These articles are not ordered chronologically, but in the contextual order of presentation in this document. My contribution to each work is described under each item.
\begin{itemize}
    \item \rapp{appendixA} - \lq\lq Magnetic dipole moment in relativistic quantum mechanics\rq\rq\ by Steinmetz et. al. (2019) is a study and comparison of DP and KGP wave equations for homogeneous magnetic fields and hydrogen-like atoms. I performed all computation, writing, and figure making in preparation of the first draft and approved the final draft before submission. I acknowledge the help and consultation of Martin Formanek (MF) and Johann Rafelski (JR) in research, writing and editing.
    \item \rapp{appendixB} - \lq\lq Strong fields and neutral particle magnetic moment dynamics\rq\rq\ by Formanek et. al. (2018) is an overview of our research group's efforts in studying neutral particle dynamics in electromagnetic fields. I wrote Section 2.1 in collaboration with MF. I consulted and helped lead author MF and co-authors Stefan Evans (SE) and Cheng Tao Yang (CTY) in editing and revising the overall manuscript.
    \item \rapp{appendixC} - \lq\lq Relativistic dynamics of point magnetic moment\rq\rq\ by Rafelski et. al. (2018) introduces a new covariant formulation of classical spin dynamics and unifies Gilbertian and Amp{\`e}rian dipoles. I wrote Section 3 in collaboration with JR and MF and aided in the computation in Section 5.1. I otherwise consulted in the research, writing, and editing process of this publication. 
    \item \rapp{appendixD} - \lq\lq A Short Survey of Matter-Antimatter Evolution in the Primordial Universe\rq\rq\ by Rafelski et. al. (2023) is a 50 page long review with many novel results describing the role of antimatter in the early universe. I, in collaboration with CTY,  supervised the document creation, combining the writing contributions of all authors (including myself, Jeremiah Birrell (JB), CTY, and JR) into one coherent presentation. I also coordinated with all authors in formatting and editing the technical figures in this review by JB, CTY, and JR.
    \item \rapp{appendixE} - \lq\lq Antimatter origin of cosmic magnetism: A first look\rq\rq\ by Steinmetz et. al. (2023) proposes a model of para-magnetization driven by the large matter-antimatter (electron-positron) content of the early universe. I carried out all writing in preparation of the first draft and approved the final draft before submission. Computation and figure making was done in collaboration with CTY who contributed key results and five technical figures. I acknowledge the help and consultation of CTY and JR in research, writing and editing.
\end{itemize}

This is not a total list of all my research efforts, but forms the basis for \rchap{chap:moment} and \rchap{chap:cosmo} of this dissertation. \rchap{chap:neutrino} contains complete but still yet unpublished work. The content in \rchap{chap:neutrino} is intended to be published shortly after the finalization of this dissertation.

I was also co-author on the following publications which are not used extensively in this dissertation and are not reprinted as appendices. They are listed in chronological order below. In these three works I consulted with MF and JR in research and editing making content clarifying contributions to these manuscripts:
\begin{itemize}
    \item \lq\lq Classical neutral point particle in linearly polarized EM plane wave field\rq\rq\ by Formanek et. al. (2019) explores the dynamical equations presented in \rapp{appendixC} for neutral particles with magnetic moment.
    \item \lq\lq Radiation reaction friction: Resistive material medium\rq\rq\ by Formanek et. al. (2020) introduces a novel model of relativistic covariant friction within a medium.
    \item \lq\lq Motion of classical charged particles with magnetic moment in external plane-wave electromagnetic fields\rq\rq\ by Formanek et. al. (2021) is a followup to the above 2019 work and \rapp{appendixC} for charged particles with magnetic moment.
\end{itemize}

%%%%%%%%%%%%%%%%%%%%%%%%%%%%%%%%%%%%%%%
\chapter{Introduction and overview}
\label{chap:intro}
%%%%%%%%%%%%%%%%%%%%%%%%%%%%%%%%%%%%%%%
This introduction serves to motivate the fundamental concepts of spin, magnetic moment and electromagnetism which have played a crucial role in the history physics and will be explored in the subsequent chapters. We discuss the relevancy of spin in physics in \rsec{sec:qspin}. Magnetic (and electric) dipoles, anomalous magnetic moments (AMM), and the wave equations which describe spin-1/2 fermions are covered in \rsec{sec:mom}. The classical connection between quantum operators, force and torque will be discussed in \rsec{sec:ehrenfest}. Lastly, \rsec{sec:flrw} covers topics in $\Lambda{\rm CDM}$ cosmology which are particular relevance to \rchap{chap:cosmo}. While the material in this chapter can largely be found elsewhere in standard literature, the presentation and organization of certain key concepts is unique. This chapter will also serve to establish notation and mathematical conventions. SI units will be used unless otherwise stated.

%%%%%%%%%%%%%%%%%%%%%%%%%%%%%%%%%%%%%%%
\section{The importance of spin}
\label{sec:qspin}
%%%%%%%%%%%%%%%%%%%%%%%%%%%%%%%%%%%%%%%
\noindent All fundamental particles known in physics have a non-zero quantized spin angular momentum with the exception of the Higgs boson which is a scalar with spin-0. All other fundamental particles (such as electrons, quarks, photons, etc...) have values of either spin-1/2 or spin-1. Particles with even values of spin are known as bosons while half-integer particles with spin are called fermions. Composite particles (such as atomic nuclei) can exhibit more exotic spin values and fundamental particles with higher spins such as spin-3/2 are commonly predicted in beyond-standard-model (BSM) physics.

In the realm of the Poincar{\'e} group of spacetime symmetry transformations (rotations, boosts and translations), each particle can be uniquely labeled by two distinct Casimir invariants (see \rsec{sec:cspin}). These two operators commute with all generators of the Poincar{\'e} group and are namely: mass and spin.

If a particle is electrically charged, then by virtue of its spin it will have a magnetic dipole moment. Most neutral particles with spin, though not all, will also have magnetic dipoles though for more complex reasons. Therefore the magnetic behavior of particle is an important window into probing one of the most fundamental properties in physics. As quantum mechanics is not well described in terms of forces or accelerations (except in the context of Ehrenfest-style equations; see \rsec{sec:ehrenfest}), there is no simple operator description of torque and spin-forces despite having played a key role in the development of quantum mechanics. For a short historical overview of spin see~\cite{ohanian1986spin}.

%%%%%%%%%%%%%%%%%%%%%%%%%%%%%%%%%%%%%%%
\section{Quantum magnetic dipoles and wave equations}
\label{sec:mom}
%%%%%%%%%%%%%%%%%%%%%%%%%%%%%%%%%%%%%%%
In classical theory, when charges rotate or circulate in some manner, a magnetic field is produced characterized by the magnetic dipole moment of the system. An Amp{\`e}rian loop of wire with a current is the quintessential example. This concept can be transplanted into quantum theory for spinning particles where the natural size of the magnetic moment of a particle (in this context a lepton) is given by the magneton value
\begin{gather}
    \label{mag:1}
    \mu_{\ell}\equiv\frac{q_{\ell}\hbar}{2m_{\ell}}
\end{gather}
where the lepton (denoted by $\ell$) has charge $q_{\ell}=\pm e$ and mass $m_{\ell}$. For electrons, this quantity is referred to as the Bohr magneton $\mu_{B}$. The non-relativistic spin operator $\bb{s}$ for a spin-1/2 particle is defined as
\begin{gather}
    \label{qspin:1}
    \bb{ s}=\frac{\hbar}{2}\bb{\sigma}=\frac{\hbar}{2}\left(\sigma_{1},\,\sigma_{2},\,\sigma_{3}\right)
\end{gather}
where $\bb{\sigma}$ is the three-vector comprised of the familiar $2\times2$ Pauli matrices.

Euclidean three-vectors and matrices will be denoted by boldface font. If indices are specifically printed, they will be done so using Latin indices such as $s_{i}$. Inner products of three-vectors will be noted via $\bb{a}\cdot\bb{b}=a_{i}b_{i}$ using Einstein summation notation where repeated indices are summed over. The algebra defined by the commutators of the Pauli matrices serves as a representation of $SU(2)$ group structure
\begin{gather}
    \label{pauli:1}
    [\sigma_{i},\sigma_{j}] = 2\varepsilon_{ijk}\sigma_{k}\,,
\end{gather}
where $\varepsilon_{ijk}$ is the totally anti-symmetric Levi-Civita symbol.

Taking inspiration from the classical (Cl.) form of magnetic dipole energy
\begin{gather}
    \label{magenergy:1}
    U=-\bb{\mu}\vert_{\rm Cl.}\cdot\bb{B}\,,
\end{gather}
we introduce the non-relativistic magnetic dipole Hamiltonian given in the usual way by
\begin{gather}
	\label{mag:2}
    {H}_{\mathrm{Mag}}=-\bb{\mu}\cdot\bb{B}\,.
\end{gather}
The magnetic moment operator $\bb{\mu}$ is defined in terms of the Pauli matrices
\begin{gather}
    \label{mag:3}
    \bb{\mu}=g\left(\frac{e\hbar}{2m_{\ell}}\right)\frac{\bb{\sigma}}{2}=g\mu_{\ell}\frac{\bb{\sigma}}{2}\,,\qquad\mu\equiv\frac{g}{2}\mu_{\ell}\,,
\end{gather}
where $\mu$ is the \lq total magneton\rq\ value representing the full magnetic moment.

In non-relativistic quantum mechanics, the time-dependant Schr{\"o}dinger-Pauli equation (with Hamiltonian $H_{\rm SP}$) for a charged particle is given by
\begin{gather}
	\label{sp:1}
    {H}_{\mathrm{SP}}\chi=\left(\frac{1}{2m_{\ell}}\bb{\pi}^{2}-\bb{\mu}\cdot\bb{B}+e{ V}\right)\chi=i\hbar\frac{\partial}{\partial t}\chi\,,\qquad
    \bb{\pi}=\bb{ p}-e{\bb{A}}\,,
\end{gather}
where $\chi$ is a two-component spinor. Spinor indices will be suppressed or noted with Latin indices. The operator $\bb{\pi}$ is the kinetic momentum operator written in terms of canonical momentum $\bb{p}$ and vector potential $\bb{A}$. The electric potential is $V$. It is well known that \req{sp:1} is obtainable from the Dirac equation (see \rsec{sec:dp}) in the non-relativistic limit.

The parameter $g$ in \req{mag:3} is the gyromagnetic ratio (or $g$-factor) of the particle. The \lq natural\rq\ value is $g=2$. While this prediction is normally attributed to the Dirac equation, it justified from the construction of the kinetic energy operator in the Schr{\"o}dinger-Pauli equation; see \rsec{sec:unique} and~\cite{sakurai1967advanced}. 

\begin{table}
	\centering
\begin{tabular}{r|c|l}
    particle & category & $g$-factor\\
    \hline
	electron & fundamental & -2.002\ 319\ 304\ 362\ 56(35)\\
	muon & fundamental & -2.002\ 331\ 8418(13)\\
	tau & fundamental & -2.036(34)\\
	neutron & composite & -3.826\ 085\ 45(90)\\
	proton & composite & \ 5.585\ 694\ 6893(16)\\
	deuterium & composite & \ 0.857\ 438\ 2338(22)\\
	tritium & composite & \ 5.957\ 924\ 931(12)\\
\end{tabular}
	\caption{The $g$-factor of various particles found in~\cite{ParticleDataGroup:2022pth}.}
	\label{tab:gfactor}
\end{table}

In nature however, there is no particle with exactly $g=2$. As seen in \rt{tab:gfactor}, composite particles often deviate from $g=2$ greatly as the $g$-factor of a composite particle is related to its internal composition. In the case of the neutron and proton, the internal quarks themselves are responsible. The comparison between three isotopes of hydrogen also displays how magnetic moments can \lq cancel out\rq. While deuterium's value of $g$ is suppressed by the extra neutron, the two neutrons in tritium balance one another returning the ratio into one manifestly similar to the proton. When $g\neq2$ (which is true for all physical particles with magnetic moment; composite of otherwise) the anomalous magnetic moment (AMM) can be defined via 
\begin{gather}
    \label{amm:1}
    a\equiv\frac{g}{2}-1\,,\qquad
    a\frac{e\hbar}{2m_{\ell}}\rightarrow\delta\mu\equiv\mu-\mu_{\ell}\,,
\end{gather}
where $a$ is the anomaly parameter. We also introduce $\delta\mu$ as the anomalous magneton which will be helpful in our proposal to connect mass and magnetic moment in \rsec{sec:ikgp} and \rsec{sec:numoment}.

The anomalous magnetic moment of a particle can arise from a variety of physical sources with the most famous being the one-loop vacuum polarization contribution to the electron first computed by~\cite{Schwinger:1951nm}. In that work, the first correction to $g$ is given by
\begin{gather}
    a_{e} = \frac{\alpha}{2\pi}\,,\qquad
    \alpha\equiv\frac{1}{4\pi\varepsilon_{0}}\frac{e^{2}}{\hbar c}\,,
\end{gather}
where $\alpha$ is the fine structure constant with an approximate value of $1/137$. The measurement of the electron's $g$-factor is among the most precise measurements in all of physics~\citep{Tiesinga:2021myr} and rapid advancements in the measurement of the muon's anomalous magnetic moment have been announced even just prior to this dissertation being finalized~\citep{Muong-2:2023cdq}. This makes the study of magnetic moment, and spin, an exciting area of physical research as new developments continue today.

%%%%%%%%%%%%%%%%%%%%%%%%%%%%%%%%%%%%%%%
\subsection{Dirac and Dirac-Pauli equations}
\label{sec:dp}
%%%%%%%%%%%%%%%%%%%%%%%%%%%%%%%%%%%%%%%
\noindent While it is always beneficial to be well-appraised of non-relativistic mechanics, nature is intrinsically relativistic and therefore this dissertation must be as well. The relativistic generalization of \req{sp:1} is the Dirac equation given by
\begin{gather}
    \label{dirac:1}
    \left(\gamma_{\alpha}\left(i\hbar\partial^{\alpha} - eA^{\alpha}\right)-m_{\ell}c\right)\psi=0\,.
\end{gather}
Four-vectors and tensors in this work will be denoted by Greek indices. Inner products of four-vectors will be noted by $a\cdot b=a^{\alpha}\eta_{\alpha\beta}b^{\beta}=a^{\alpha}b_{\alpha}$ again following Einstein notation. The four-derivative $\partial^{\alpha}$ and four-potential $A^{\alpha}$ are defined as
\begin{gather}
    \label{dirac:2}
    \partial^{\alpha}=\left(\frac{1}{c}\frac{\partial}{\partial t},\,-\bb{\nabla}\right)\,,\qquad A^{\alpha}=\left(\frac{V}{c},\,\bb{A}\right)\,.
\end{gather}
The wave function $\psi$ in \req{dirac:1} is understood to be a four-component spinor. We have written the Dirac equation here in the covariant form where $\gamma^{\alpha}$ are the gamma matrices which obey the anti-commuting Clifford algebra
\begin{gather}
    \label{gamma:1}
    \{\gamma_{\alpha},\gamma_{\beta}\}=\gamma_{\alpha}\gamma_{\beta} + \gamma_{\beta}\gamma_{\alpha} = 2\eta_{\alpha\beta}\,,\\
    \eta_{\alpha\beta}={\rm diag}(+1,-1,-1,-1)\,,
\end{gather}
where $\eta_{\alpha\beta}$ is the flat spacetime Minkowski metric tensor defined with a positive time metric signature. The metric tensor is also responsible for raising and lowering covariant and contravariant indices e.g. $a_{\alpha}=\eta_{\alpha\beta}a^{\beta}$. As $\gamma^{\alpha}$ are also spinor matrices, the commutator in \req{gamma:1} carries implicit spinor indices which here computes to the $4\times4$ identity matrix $\mathbbm{1}_{4}$ (which is suppressed). We also introduce the \lq fifth\rq\ gamma matrix $\gamma^{5}$ which anti-commutes with $\gamma^{\alpha}$ and the following standard conventions following~\cite{Itzykson:1980rh}
\begin{alignat}{1}
	\label{conventions:1} \bb{\alpha}=\gamma^{0}\bb{\gamma}\,,\indent \bb{\Sigma}=\gamma^{5}\bb{\alpha}\,,\indent \gamma^{5}=i\gamma^{0}\gamma^{1}\gamma^{2}\gamma^{3}\,,\indent \gamma^{2}_{5}=1\,.
\end{alignat}

As mentioned before, \req{dirac:1} predicts $g=2$ which is a standard calculation in many textbooks. The most straight-forward manner to generalize the Dirac equation allowing for an anomalous magnetic moment is to add a Pauli term proportional to the anomalous parameter $a$. While in most texts, the anomaly is given in terms of $g-2$ or $a$, we wish to keep our equations generalized to fermions of any given charge $e$ and magnetic moment $\mu$. 

Therefore we make use of the substitution in \req{amm:1} and write the Dirac-Pauli~(DP) equation as
\begin{gather}
	\label{dp:1}
    \left(\gamma_{\alpha}\left(i\hbar\partial^{\alpha} - eA^{\alpha}\right) - m_{\ell}c - \delta\mu\frac{1}{2c}\sigma_{\alpha\beta}F^{\alpha\beta}\right)\psi=0\,,
\end{gather}
where the anti-symmetric spin tensor $\sigma_{\alpha\beta}$ is defined in terms of the commutator of the gamma matrices
\begin{alignat}{1}
	\label{sigma:1} \sigma_{\alpha\beta}=\frac{i}{2}\left[\gamma_{\alpha},\gamma_{\beta}\right]=\frac{i}{2}\left(\gamma_{\alpha}\gamma_{\beta}-\gamma_{\beta}\gamma_{\alpha}\right)\,.
\end{alignat}
Exact solutions to the DP equation are relatively scarce due to the complicating nature of the anomalous term. The most extensively studied solutions are those with high symmetries or constant external fields \citep{Thaller:1992ji}. When the anomalous part $\delta\mu$ is zero, the Dirac equation is recovered. $F^{\alpha\beta}$ is the standard anti-symmetric electromagnetic field tensor defined by
\begin{gather}
    \label{em:1}
    F^{\alpha\beta} = \partial^{\alpha}A^{\beta} - \partial^{\beta}A^{\alpha} = 
    \begin{pmatrix}
        0        & -E_{1}/c  & -E_{2}/c  & -E_{3}/c\\
        E_{1}/c  & 0         & -B_{3}    & B_{2}\\
        E_{2}/c  & B_{3}     & 0         & -B_{1}\\
        E_{3}/c  & -B_{2}    & B_{1}     & 0
    \end{pmatrix}\,.
\end{gather}
It is also useful to define the Hodge dual of the electromagnetic field tensor
\begin{gather}
    \label{em:2}
    F_{\alpha\beta}^{*} = \frac{1}{2}\varepsilon_{\alpha\beta\mu\nu}F^{\mu\nu} = 
    \begin{pmatrix}
        0        & -B_{1}  & -B_{2}  & -B_{3}\\
        B_{1}  & 0         & -E_{3}/c    & E_{2}/c\\
        B_{2}  & E_{3}/c     & 0         & -E_{1}/c\\
        B_{3}  & -E_{2}/c    & E_{1}/c     & 0
    \end{pmatrix}\,,
\end{gather}
where we use the four-dimensional Levi-Civita pseudo-tensor $\varepsilon_{\alpha\beta\mu\nu}$ with the $\varepsilon_{0123}=+1$ convention. The contracted portion $\sigma_{\alpha\beta}F^{\alpha\beta}$ in the Pauli term in \req{dp:1} can be further expressed as
\begin{alignat}{1}
	\label{dp:2} \frac{1}{2}\sigma_{\alpha\beta}F^{\alpha\beta}=\frac{i}{c}\bb{\alpha}\cdot\bb{E}-\bb{\Sigma}\cdot\bb{B}\,,
\end{alignat}
which captures that relativistic magnetic moments should be sensitive to electric as well as magnetic fields. This should be unsurprising if one considers how the non-relativistic dipole form must generalize under Lorentz boost.

The DP equation can be derived as an effective wave equation arising from the perturbative vacuum contribution developed by Schwinger~\citep{Itzykson:1980rh,Schwartz:2014sze} which modifies the photon 3-vertex in quantum electrodynamics. Therefore we can describe the AMM as an added Lagrangian interaction term
\begin{gather}
    \label{lamm:1}
    {\cal L}_{\rm AMM} = -{\bar\psi}\left(\delta\mu\frac{1}{2}\sigma_{\alpha\beta}F^{\alpha\beta}\right)\psi\,,
\end{gather}
where ${\bar\psi}=\psi^{\dagger}\gamma^{0}$ is the Dirac adjoint. While the focus of this dissertation is not on quantum field theory (QFT), it is valuable to note that the Pauli Lagrangian term in \req{lamm:1} is not renormalizable as the coupling coefficient has a natural unit of length.

%%%%%%%%%%%%%%%%%%%%%%%%%%%%%%%%%%%%%%%
\subsubsection{Electric dipole moments}
\label{sec:edm}
%%%%%%%%%%%%%%%%%%%%%%%%%%%%%%%%%%%%%%%
\noindent While this dissertation is primarily concerned with the dynamics of magnetic dipoles, we note that the techniques and methods discussed here may also find application in other important topics such as electric dipoles. I stress that the following section should be treated as an interesting avenue for future, not present, work.

The structure of the Pauli term in \req{lamm:1} informs us how to construct the relativistic electric dipole moment (EDM); see~\cite{Knecht:2003kc,Jegerlehner:2017gek}. The generalization to include the electric dipole is
\begin{alignat}{1}
	\label{edm:1} \delta\mu\rightarrow\delta\tilde{\mu}\equiv\delta\mu+i\epsilon\gamma^{5}\,,
\end{alignat}
where $\epsilon$ is the EDM of the particle. As the natural electric dipole within the Dirac equation is zero, the presence of $\epsilon$ is always considered anomalous. The EDM Pauli Lagrangian term is
\begin{gather}
    \label{ledm:1}
    {\cal L}_{\rm EDM} = -{\bar\psi}\left(i\epsilon\gamma^{5}\frac{1}{2}\sigma_{\alpha\beta}F^{\alpha\beta}\right)\psi\,,
\end{gather}
which is of interest because of the inclusion of $\gamma^{5}$. Taking advantage of the properties of $\gamma^{5}$, we can write the EDM in \req{ledm:1} as
\begin{gather}
    \label{ledm:3}
    \gamma^{5}\sigma_{\mu\nu}=\frac{i}{2}\varepsilon_{\mu\nu\alpha\beta}\sigma^{\alpha\beta}\,\rightarrow
    {\cal L}_{\rm EDM} = +{\bar\psi}\left(\epsilon\frac{1}{2}\sigma^{\alpha\beta}F_{\alpha\beta}^{*}\right)\psi\,.
\end{gather}
which is more closely analogous to the structure of the AMM in \req{lamm:1} making use of the dual form of the electromagnetic field tensor shown in \req{em:2}.

The electric dipole is important because the presence of one would signify CP violation in the theory (assuming CPT is a true symmetry). Electrical fields are T-even while While such dipoles are common in composite or molecular systems, no electric dipole has ever been measured for a fundamental particle. As a point of comparison, the EDM of the electron is excluded~\citep{ACME:2018yjb,Roussy:2022cmp} by a bound of $|\epsilon_{e}/c|<4.1\times10^{-30}\, e\,{\rm cm}$.

%%%%%%%%%%%%%%%%%%%%%%%%%%%%%%%%%%%%%%%
\subsection{Klein-Gordon-Pauli equation}
\label{sec:kgp}
%%%%%%%%%%%%%%%%%%%%%%%%%%%%%%%%%%%%%%%
\noindent While the DP equation is more commonly used, there exists an alternative wave equation which describes the magnetic behavior of fermions called the Klein-Gordon-Pauli (KGP) equation. This equation was first introduced by~\cite{Fock:1937dy} and found usefulness in studying weak interactions~\citep{Feynman:1958ty} due to the ease of describing chiral states. This equation is physically distinct from the DP and Dirac equations and only share solutions when $g=2$.

The KGP equation is generally considered to be the \lq\lq square\rq\rq\ of the Dirac equation as unlike the Dirac or DP equations, it is a second order equation wave equation for the four-component spinor $\Psi$
\begin{alignat}{1}
	\label{kgp:1} \left((i\hbar\partial^{\alpha}-eA^{\alpha})^{2}-m_{\ell}^{2}c^{2}-\left(g\mu_{\ell}+i\epsilon\gamma^{5}\right)m_{\ell}\frac{1}{2}\sigma_{\alpha\beta}F^{\alpha\beta}\right)\Psi=0\,.
\end{alignat}
In the above we printed both the AMM and EDM for theoretical interest, but will drop the EDM term for the remainder of this dissertation. The initial benefit of the KGP formulation is that the wave equation fully commutes with $\gamma^{5}$ making eigen-functions explicitly good chiral states.

\req{kgp:1} is mathematically similar to the Klein-Gordon equation which describes charged scalar particles. In the same manner as scalar-QED, the squared covariant derivative contains a $e^{2}A^{2}$ term which in QFT results in the presence of a 4-vertex seagull interaction~\citep{Schwartz:2014sze} at tree-level.

It is important to emphasize that the KGP \req{kgp:1} and DP \req{dp:1} are distinct wave equations which do not share solutions except when $g=2$ and both reduce to the Dirac \req{dirac:1}. We will elaborate on the relationship between the KGP and Dirac questions here by rewriting the Dirac equation in \req{dirac:1} as
\begin{alignat}{1}
	\label{do:1} \mathcal{D}_{\pm}=\gamma_{\alpha}(i\hbar\partial^{\alpha}-eA^{\alpha})\pm mc\,,\qquad
    {\cal D}_{-}\psi=0\,,
\end{alignat}
with a \lq Dirac operator\rq\ defined in terms of $\pm m$ and the following properties
\begin{gather}
    \label{do:2}
    \mathcal{D}_{-}=-\gamma^{5}\mathcal{D}_{+}\gamma^{5}\,,\qquad
    [{\cal D}_{+},{\cal D}_{-}]=0\,.
\end{gather}
We can complete the square of the Dirac equation via the substitution
\begin{alignat}{1}
	\label{eq:kgp:03} \mathcal{D}_{+}\psi\rightarrow\mathcal{D}_{+}\mathcal{D}_{-}\Psi\,,\indent\mathcal{D}_{+}\mathcal{D}_{-}\Psi=\left((i\hbar\partial-eA)^{2}-m^{2}c^{2}-\frac{e\hbar}{2}\sigma_{\alpha\beta}F^{\alpha\beta}\right)\Psi=0\,.
\end{alignat}
This procedure yields the KGP equation for $g=2$. More generally however, the KGP equation is able to describe fermions of any arbitrary moment.

%%%%%%%%%%%%%%%%%%%%%%%%%%%%%%%%%%%%%%%
\subsubsection{Technical quibbles with magnetic moment Lagrangian}
The Dirac-Pauli equation can be obtained from perturbative QED as an effective field theory for leptons due to vacuum polarization; see standard texts \cite{Itzykson:1980rh,Schwartz:2014sze}. However, if a particle's anomalous magnetic moment is not sourced by perturbative QFT, then the Pauli term introduced in \rsec{sec:dp} must be added by hand \emph{ad hoc} or obtained via non-perturbative means such as Lattice calculations~\citep{Aoyama:2020ynm}. This is the case for the hadronic contribution to anomalous magnetic moment of leptons as well as any composite particle such as the proton or neutron whose moment is determined by internal structure~\citep{Proceedings:2012ulb}.

As such, there is no reason to expect non-perturbative sources of magnetic moment to strictly adhere to the DP form. Additionally, the DP equation has the physically inelegant consequence of splitting the spin dynamics of fermions into a natural behavior encompassed by the spinor structure of the Dirac equation and the anomalous behavior contained in the Pauli term. This is not the case for the KGP equation or Dirac equations where the magnetic moment is entirely described by the same mathematical formalism of a Pauli term in the case of KGP and the spinor structure of the gamma matrices in the case of the Dirac equation.

As a quantum field theory, both DP and KGP run into difficulties, though each inherits a different problem. The Lagrangian which produces the DP equation is identical to the Dirac Lagrangian except for the addition of the anomalous interaction
\begin{alignat}{1}
	\label{eq:problems01} \mathcal{L}_{\mathrm{DP,AMM}}=-a\mu_{\ell}\bar{\psi}\frac{1}{2}\sigma_{\alpha\beta}F^{\alpha\beta}\psi\,.
\end{alignat}
This Lagrangian interaction has a coefficient of $[length]^{1}$ in natural units which makes it unsuitable for renormalization which is an essential feature required for the QFT to describe differing energy scales. While this does not stop us using DP as an effective QFT with some natural cutoff scale responsible for the anomalous moment, it does reduce the usefulness of the equation as a general description of quantum dipole moments.

The KGP equation can be obtained from a Lagrangian not dissimilar to the Klein-Gordon Lagrangian and has the expression
\begin{alignat}{1}
	\label{eq:problems02} \mathcal{L}_{\mathrm{KGP}}=\hbar^{2}c^{2}\left(\nabla^{\alpha}_{\mathrm{A}}\bar{\Psi}\right)h_{\alpha\beta}\left(\nabla^{\beta}_{\mathrm{A}}\Psi\right)-m^{2}c^{4}\bar{\Psi}\Psi\,.
\end{alignat}
For $g\neq2$ the relationship between the DP and KGP equation becomes more complicated. Instead of a simpler root, the linearized KGP Lagrangian requires an infinite series expansion \citep{Veltman:1997am} resulting from the non-local inverse substitution 
\begin{alignat}{1}
	\label{eq:kgp:04} \Psi\rightarrow\frac{1}{\mathcal{D}_{-}}\psi=\frac{1}{i\hbar\gamma_{\alpha}\partial^{\alpha}-m_{\ell}c}\left(1-\frac{1}{i\hbar\gamma_{\alpha}\partial^{\alpha}-m_{\ell}c}e\gamma_{\alpha}A^{\alpha}+\ldots\right)\psi\,.
\end{alignat}
This non-local behavior ultimately breaks the unitarity of the theory making it also unsuitable as a fundamental particle theory. If a generalized description of magnetic moment outside $g=2$ theories exists and makes a good fundamental quantum field theory, then likely non-minimal electromagnetic terms are required to maintain both renormalization and unitarity.

%{\color{red} Thought: What about ``strong field case'' where $A^{\alpha}$ is very large, even compared to the momentum states, would the expansion yield a nicer result? In otherwords, can strong fields restore unitarity?}

%{\color{red}Here we talk about some of the problems DP has. The theory is non-renormalizable. It spits the magnetic behavior into two very different mathematical regimes. The QED vacuum contribution to the electron \lq\lq melts\rq\rq\ in strong fields (though this might not affect the hadronic component) making DP it unsuitable. This last point is a light criticism of Ferrer and astrophysics community.}

%%%%%%%%%%%%%%%%%%%%%%%%%%%%%%%%%%%%%%%
\subsection{The unique nature of g=2}
\label{sec:unique}
%%%%%%%%%%%%%%%%%%%%%%%%%%%%%%%%%%%%%%%
There is a strong predilection in nature towards $g=2$ which can be explained by the requirements of kinetic operator in quantum mechanics. We note that because the $2\times2$ Pauli matrices $\bb{\sigma}$ all anti-commute, we can write down the relation
\begin{alignat}{1}
	\label{nat:1} (\bb{\sigma}\cdot\bb{a})(\bb{\sigma}\cdot\bb{b})=\bb{a}\cdot\bb{b}+i\bb{\sigma}\cdot(\bb{a}\times\bb{b})\,.
\end{alignat}
The non-relativistic kinetic energy Hamiltonian from \req{sp:1} then reads as
\begin{alignat}{1}
	\label{nat:2} {H}_{\mathrm{KE}}=\frac{1}{2m}\left(\bb{\sigma}\cdot\bb{\pi}\right)^{2}=\frac{1}{2m}\bb{\pi}^{2}+i\bb{\sigma}\cdot(\bb{\pi}\times\bb{\pi})\,.
\end{alignat}
As the kinetic momentum operator $\bb{\pi}$ does not self-commute, its cross-product is non-zero resulting in
\begin{alignat}{1}
	\label{nat:3} {H}_{\mathrm{KE}}=\frac{1}{2m}\bb{\pi}^{2}-\frac{e\hbar}{2m}\bb{\sigma}\cdot\bb{B}\,,
\end{alignat}
which is the magnetic moment term with size $g=2$. We recognize that $g=2$ appears to arise from the $\mathfrak{su}(2)$ Lie algebra (Schultz, 1980) representation described by the Pauli matrices and electromagnetic minimal coupling. The natural scale of the magnetic moment can be interpreted as originating from group symmetry requirements on charged particles. Rather than taking the non-relativistic limit of the Dirac equation, $g=2$ can also be derived as a consequence of replacing the definition of the inner product for vectors which accounts for spin (Feynman, 1961). The Schrodinger kinetic Hamiltonian and Schrodinger-Pauli kinetic terms are then related via substitution
\begin{alignat}{1}
	\label{nat:4} \bb{\pi}\cdot\bb{\pi} \rightarrow (\bb{\pi}\cdot\bb{\sigma})(\bb{\sigma}\cdot\bb{\pi})\,.
\end{alignat}
An almost identical argument that $g$-factor arises from spin-structure and electromagnetic coupling can be made for the relativistic case as well. First we consider the quantum analog to the energy-momentum relation
\begin{alignat}{1}
	\label{analog:1} \eta_{\alpha\beta}p^{\alpha}p^{\beta}\Psi=m^{2}c^{2}\Psi\,.
\end{alignat}
\req{analog:1} as written evaluates to the Klein-Gordon equation when the four-momentum is written in the position basis $p^{\alpha}\rightarrow i\hbar\partial^{\alpha}$. Much like the non-relativistic example, we can introduce spin by replacing the momentum inner product with one sensitive to a Clifford algebra (Weinberg, 1995). Rather than the Pauli matrices, the relativistic replacement utilizes the gamma matrices $\eta_{\alpha\beta}\rightarrow\gamma_{\alpha}\gamma_{\beta}$ yielding
\begin{alignat}{1}
	\label{eq:spin:03} \gamma_{\alpha}\gamma_{\beta}p^{\alpha}p^{\beta}\Psi&=m^{2}c^{2}\Psi\,.
\end{alignat}
Here $\Psi$ is understood to be a four-component spinor unlike in \req{analog:1}. The corresponding $4\times4$ matrix contraction identity analog to \req{nat:1} is then
\begin{alignat}{1}
	\label{eq:spin:04} \gamma_{\alpha}\gamma_{\beta}a^{\alpha}b^{\beta}=\eta_{\alpha\beta}a^{\alpha}b^{\beta}-i\sigma_{\alpha\beta}a^{\alpha}b^{\beta}\,,\indent \sigma_{\alpha\beta}\equiv\frac{i}{2}\left[\gamma_{\alpha},\gamma_{\beta}\right]\,.
\end{alignat}
The anti-symmetric spin tensor $\sigma_{\alpha\beta}$ is defined by the commutator of the gamma matrices. In both the relativistic and non-relativistic cases, the distinction between spin-1/2 and spinless particles is only made kinematically apparent in the presence of electromagnetic fields. For minimal coupling
\begin{alignat}{1}
  \label{eq:spin:05} \pi^{\alpha}=p^{\alpha}-eA^{\alpha}\,,
\end{alignat}
we take advantage of the fact that any product can be written as a sum of commuting and anti-commuting parts
\begin{alignat}{1}
	\label{eq:spin:06} \pi^{\alpha}\pi^{\beta}=\frac{1}{2}\left(\left\{\pi^{\alpha},\pi^{\beta}\right\}+\left[\pi^{\alpha},\pi^{\beta}\right]\right)\,.
\end{alignat}
yielding
\begin{alignat}{1}
	\label{eq:spin:07a} \left(\eta_{\alpha\beta}\pi^{\alpha}\pi^{\beta}-\frac{i}{2}\sigma_{\alpha\beta}\left[\pi^{\alpha},\pi^{\beta}\right]\right)\Psi&=m^{2}c^{2}\Psi\,,\\
	\label{eq:spin:07b} \left(\eta_{\alpha\beta}\pi^{\alpha}\pi^{\beta}-\frac{e\hbar}{2}\sigma_{\alpha\beta}F^{\alpha\beta}\right)\Psi&=m^{2}c^{2}\Psi\,.
\end{alignat}
\req{eq:spin:07b} is the square of the Dirac equation with precisely $g=2$. Furthermore, compelling arguments can be made that all elementary particles (Ferrara et. al. 1992) of any spin have a natural value of $g=2$, though a competing idea is Belinfante's conjecture of $g=1/s$. To paraphrase the argument by~\citet*{Ferrara:1992yc}, $g=2$ is likely the natural scale for particles of any spin because:
\begin{enumerate}[nosep]
	\item The W boson, as the only known higher spin charged elementary particle, has at tree level $g=2$ via a Proca-like equation.
	\item The relativistic TBMT spin equation is the same simplified form for any classical spin value.
	\item For arbitrary spin, $g=2$ facilitates finite Compton scattering cross sections without additional physical requirements.
	\item For arbitrary spin, open bosonic and super-symmetric string theory predicts $g=2$.
\end{enumerate}
While the above provide a nice justification for why particles should tend to this specific $g$-factor, the reality is no particle has exactly $g=2$ with all of them displaying some form of anomaly. The charged leptons come the closest to the natural value, but famously have vacuum polarization contributions~\citep{Schwinger:1951nm} from QED, non-perturbative hadronic contributions~\citep{Jegerlehner:2017gek}, and potentially BSM interactions~\citep{Knecht:2003kc} contributing to their anomalous magnetic dipole moment.

We then outline the two main approaches to anomalous moments: The first is to consider the fundamental quantum field theory describing the particle, and then analyzing the resulting tree level diagrams which contribute to the moment in a perturbative fashion. The second is to utilize an effective field theory with the anomalous moment already implemented in the base wave dynamics. While the first approach has proven to be exceedingly successful for the charged leptons (the muon's anomaly notwithstanding), it is not appropriate for particles whose moments are dramatically different from $g=2$ or if the origin of the anomaly comes from internal structure such as the hadrons whose moments are determined by non-perturbative QCD~\citep{Eichmann:2016yit,Pacetti:2014jai} and not vacuum structure as is the case for the nucleons and atoms.

%%%%%%%%%%%%%%%%%%%%%%%%%%%%%%%%%%%%%%%
\section{Connecting quantum and classical theory}
\label{sec:ehrenfest}
%%%%%%%%%%%%%%%%%%%%%%%%%%%%%%%%%%%%%%%
\noindent The relativistic precession of a spin-1/2 particle can be extracted from the Dirac equation. The Dirac equation with electromagnetic interaction is given by
\begin{alignat}{1}
  \label{DIRAC01} i\hbar\frac{\partial}{\partial t}\psi=\left(\boldsymbol{\alpha}\cdot\left(\mathbf{p}c-e\mathbf{A}\right)+eA_{0}+\gamma^{0}mc^{2}\right)\psi=\hat{H}_{D}\psi\,,
\end{alignat}

In the Heisenberg representation, operators obeys the following equation of motion
\begin{alignat}{1}
  \label{m06}		&i\hbar\frac{\mathrm{d}\hat{O}}{\mathrm{d}t}-i\hbar\frac{\partial\hat{O}}{\partial t}=[\hat{O},\hat{H}]
\end{alignat}
and the expectation value is of course
\begin{alignat}{1}
  \label{m07}		&\langle i\hbar\frac{\mathrm{d}\hat{O}}{\mathrm{d}t}\rangle-\langle i\hbar\frac{\partial\hat{O}}{\partial t}\rangle=\langle[\hat{O},\hat{H}]\rangle
\end{alignat}

The operator spin precession is
\begin{alignat}{1}
  \label{DIRAC04} \frac{\mathrm{d}\mathbf{S}_{R}}{\mathrm{d}t}=\frac{\mathrm{d}}{\mathrm{d}t}\left(\frac{\hbar}{2}\boldsymbol{\Sigma}\right)=-\boldsymbol{\alpha}\times\left(\mathbf{p}c-e\mathbf{A}\right)\,,
\end{alignat}
where $\mathbf{S}_{R}$ is the relativistic spin operator. While this expression is fully relativistic, it is not easily interpreted due to the off-diagonal nature of the $\boldsymbol{\alpha}$ matrices which mix particle and antiparticle components. A similar situation arises for the \lq\lq velocity operator\rq\rq\, which leads to the phenomenon of \emph{Zitterbewegung}. Eq.~\eqref{DIRAC04} reveals that spin is not a constant of motion of the Dirac equation. The orbital angular momentum $\mathbf{L}$ is also not a constant of motion, but the sum of the two or the total angular momentum $\mathbf{J}$ is.

The expression for the Lorentz force derivable from the Schr{\"o}dinger-Pauli equation is given by
\begin{alignat}{1}
  \label{m31}		&\frac{\mathrm{d}\pi_{k}}{\mathrm{d}t}=eE_{k}+\frac{e}{2mc}(\vec{\pi}\times\vec{B}-\vec{B}\times\vec{\pi})_{k}+\frac{e\hbar}{2mc}\vec{\sigma}\cdot\nabla_{k}\vec{B}
\end{alignat}
which contains the Stern-Gerlach force and correct magnetic moment with $g=2$.

%%%%%%%%%%%%%%%%%%%%%%%%%%%%%%%%%%%%%%%
\subsubsection{Foldy-Wouthuysen transformation}
%%%%%%%%%%%%%%%%%%%%%%%%%%%%%%%%%%%%%%%
To connect the relativistic quantum notion of spin to our non-relativistic classical intuition we need to disentangle the components of the Dirac equation. To do this we make use of the Foldy-Wouthuysen (FW) transformation which diagonalizes odd operators via a unitary transformation
\begin{alignat}{1}
  \label{FW01} \psi\rightarrow\psi'=e^{i\mathcal{S}}\psi,\indent\hat{H}\rightarrow\hat{H}'=e^{i\mathcal{S}}\hat{H}e^{-i\mathcal{S}}-ie^{i\mathcal{S}}\frac{\partial e^{-i\mathcal{S}}}{\partial t}\,,
\end{alignat}
where $\mathcal{S}$ is Hermitian. The resulting Hamiltonian $\hat{H}'$ then produces two uncoupled equations of two-component spinors. For free particles the transformed Hamiltonian has an exact closed form, but for charged particles in weak fields the Hamiltonian will be an infinite series in powers of inverse mass $1/m$. Our interest lies in determining the first nonlinear terms in the spin precession of order $|\mathbf{s}|^{2}$ as these are the first terms to be sensitive to field in-homogeneity. Because of the intimate relationship between spin and magnetic moment, the expressions we seek will be up to order $\mu^{2}\sim e^{2}/m^{2}$ and thus we only require the FW transformed Hamiltonian up to $1/m^{2}$.

The generated Hamiltonian is therefore
\begin{alignat}{1}
  \notag\hat{H}'&=\gamma^{0}\left\{mc^{2}+\frac{1}{2m}\left(\mathbf{p}-\frac{e}{c}\mathbf{A}\right)^{2}-eA_{0}-\frac{e\hbar}{2mc}\boldsymbol{\sigma}\cdot\mathbf{B}\right.\\
  \notag&+\frac{e\hbar}{8m^{2}c^{2}}\boldsymbol{\sigma}\cdot\left(\left(\mathbf{p}-\frac{e}{c}\mathbf{A}\right)\times\mathbf{E}-\mathbf{E}\times\left(\mathbf{p}-\frac{e}{c}\mathbf{A}\right)\right)\\
  \label{FW02}&-\left.\frac{e\hbar^{2}}{8m^{2}c^{2}}\boldsymbol{\nabla}\cdot\mathbf{E}+\ldots\right\}\,.
\end{alignat}
The $\boldsymbol{\sigma}$ operators are simply the Pauli matrices. The first four terms represent the rest mass-energy of the particle and the Schr\"{o}dinger-Pauli (SP) Hamiltonian, the fifth term is the spin-orbit coupling, and the last is the Darwin term. In determining the precession, it is the spin Hamiltonian
\begin{alignat}{1}
  \notag\hat{H}'_{spin}=&-\frac{e\hbar}{2mc}\boldsymbol{\sigma}\cdot\bigg(\mathbf{B}\\
  \label{FW03}&-\frac{1}{4mc}\left(\left(\mathbf{p}-\frac{e}{c}\mathbf{A}\right)\times\mathbf{E}-\mathbf{E}\times\left(\mathbf{p}-\frac{e}{c}\mathbf{A}\right)\right)\bigg),
\end{alignat}
which will be relevant.

%%%%%%%%%%%%%%%%%%%%%%%%%%%%%%%%%%%%%%%
\subsection{Second order spin effects in the Stern-Garlach force}
\label{sec:spinspin}
%%%%%%%%%%%%%%%%%%%%%%%%%%%%%%%%%%%%%%%
By substituting eq.~\eqref{FW03} into eq.~\eqref{DIRAC03}, the non-relativistic spin precession (up to order $1/m^{2}$) is found to be
\begin{alignat}{1}
  \notag\frac{\mathrm{d}\mathbf{S}_{NR}}{\mathrm{d}t}=\frac{\mathrm{d}}{\mathrm{d}t}\left(\frac{\hbar}{2}\boldsymbol{\sigma}\right)&=\frac{e\hbar}{2mc}\boldsymbol{\sigma}\times\mathbf{B}\\
  \notag&-\frac{e\hbar}{4m^{2}c^{2}}\boldsymbol{\sigma}\times\left(\mathbf{E}\times\left(\mathbf{p}-\frac{e}{c}\mathbf{A}\right)\right)\\
  \label{SPIN01}&-\frac{e\hbar^{2}}{16m^{2}c^{2}}\left[\boldsymbol{\sigma},\boldsymbol{\sigma}\cdot\left(\boldsymbol{\nabla}\times\mathbf{E}\right)\right]\,,
\end{alignat}
where $\mathbf{S}_{NR}$ is the non-relativistic spin operator. The first term is the expected Larmor precession, and the second is precession generated by spin-orbit coupling. The last term is second order in spin and sensitive to the curl of electric fields. This term cannot be easily categorized as a simple \lq\lq quantum correction\rq\rq\, as both orders of $\hbar$ are consumed by the definition of the spin operator and survive the $\hbar\rightarrow 0$ limit. This term in fact is more analogous to the $p^{4}$ correction to a particle's energy due to relativity.

As spin dynamics in quantum mechanics exhibits more complicated behavior at higher orders, due to special relativity, we expect similar corrections to appear in the classical theory of spin precession in the same manner that energy corrections appear.

%%%%%%%%%%%%%%%%%%%%%%%%%%%%%%%%%%%%%%%
\section{Feature of FLRW in Lambda-CDM}
\label{sec:flrw}
%%%%%%%%%%%%%%%%%%%%%%%%%%%%%%%%%%%%%%%
\noindent This section introduces some necessary concepts which will be useful in describing the magnetization of the electron-positron primordial plasma in \rchap{chap:cosmo}. We operate under the $\Lambda{\rm CDM}$ model of cosmology where the contemporary universe is approximately 69\% dark energy, 26\% dark matter, 5\% baryons, and $<1$\% photons and neutrinos in energy density~\citep{Planck:2018vyg}. The spatially flat (with Gaussian curvature $k=0$) Friedmann-Lema{\^i}tre-Robertson-Walker (FLRW) line element and metric~\citep{weinberg1972gravitation} in Cartesian coordinates is
\begin{gather}
    \label{FLRW} ds^2=dt^2-a^2(t)\left[dx^2+dy^2+dz^2\right]\,,\\
    g_{\alpha\beta}=
    \begin{pmatrix}
        1&0&0&0\\
        0&-a^{2}(t)&0&0\\
        0&0&-a^{2}(t)&0\\
        0&0&0&-a^{2}(t)
    \end{pmatrix}\,.
\end{gather}
The scale factor $a(t)$ denotes the change of proper distances $L(t)$~\citep{Davis:2003ad} over time as
\begin{gather}
    L(t)=L_{0}\frac{a(t_{0})}{a(t)}\rightarrow L(z)=L_{0}(1+z)\,,
\end{gather}
where $z$ is the redshift and $L_{0}$ the comoving length. In an expanding (or contracting) universe which is both homogeneous and isotropic, this implies volumes change with $\propto1/a^{3}(t)$. The evolutionary expansion of the universe is then traditionally defined in terms of the Hubble parameter $H(t)$ as follows
\begin{gather}
  \label{Friedmann} H(t)^{2}\equiv\left(\frac{\dot a}{a}\right)^2=\frac{8\pi G_{N}}{3}\rho_{\rm total},\qquad \rho_{\rm total}(t)=\rho_{\Lambda}+\rho_{\rm DM}(t)+\rho_{\rm Baryons}(t)+\ldots\\
  \frac{\ddot a}{a}=-qH^2,\qquad 
q\equiv -\frac{a\ddot a}{\dot a^2},\qquad \dot H=-H^2(1+q).
\end{gather}
where $G_N$ is the Newtonian constant of gravitation. \req{Friedmann} is also known as the Friedmann equations. The total density $\rho_{\rm total}$ is the sum of all contributions from any form of matter, radiation or field. This includes but is not limited to: dark energy $(\Lambda)$, dark matter (DM), baryons (B), and photons $(\gamma)$. Depending on the age of the universe, the relative importance of each group changes as each dilutes different under expansion with dark energy infamously remaining constant in density and accelerating the universe today.

The parameter $q$ is the cosmic deceleration parameter where for historical reasons expansion is slowing down for $q>0$. before the discovery of dark energy. The early universe was radiation dominated with $q = 1$, subsequently matter dominated with $q = 1/2$, and the contemporary universe is undergoing a transition from matter to dark energy dominated whereas the deceleration settles on the asymptotic value of $q = -1$; see~\cite{Rafelski:2013yka}.

We can consider the expansion to be an adiabatic process~\citep{Abdalla:2022yfr} which results in a smooth shifting of the relevant dynamical quantities. As the universe undergoes isotropic expansion, the temperature decreases and conserves entropy as 
\begin{gather}
 \label{tscale}
 T(t)=T_{0}\frac{a(t_{0})}{a(t)}\rightarrow T(z)=T_{0}(1+z)\,,
\end{gather}
where $z$ is the redshift. The comoving temperature $T_{0}$ is given by the present day temperature of the CMB, with contemporary scale factor $a_{0}=1$.

Within a homogeneous magnetic domain defined along the $z$-axis
\begin{gather}
    \label{homoB:1}
    \bb{B}=(0,\,0,\,B)\,,
\end{gather}
the magnetic field magnitude varies~\citep{Durrer:2013pga} over cosmic expansion as
\begin{gather}
 \label{bscale}
 B(t)=B_{0}\frac{a_{0}^{2}}{a^{2}(t)}\rightarrow B(z)=B_{0}\left(1+z\right)^{2}\,,
\end{gather}
where $B_{0}$ is the comoving value of the magnetic field obtained from the contemporary value of the magnetic field today given in \req{igmf}. Non-primordial magnetic fields (which are generated through other mechanisms such as dynamo or astrophysical sources) do not follow this scaling~\citep{Pomakov:2022cem}. The presence of matter and late universe structure formation also contaminates the primordial field evolution in \req{bscale}. It is only in deep intergalactic space where primordial fields remain preserved and comoving over cosmic time.

From the conservation of magnetic flux through a co-moving surface, the magnetic field under expansion starting at some initial time $t_{0}$ is given by
\begin{alignat}{1}
    \label{BScale} B(t) = B(t_{0})\frac{a(t_{0})^{2}}{a(t)^{2}}\,.
\end{alignat}
As the universe expands, the temperature also cools as the cosmological redshift reduces the momenta of particles in the universe lowering their contribution to the energy content of the universe. This cosmological redshift is written as
\begin{alignat}{1}
  \label{Redshift} p_{i}(t) = p_{i}(t_{0})\frac{a(t_{0})}{a(t)}\,,\qquad T(t) = T(t_{0})\frac{a(t_{0})}{a(t)}\,.
\end{alignat}
The momenta scale with the same factor as temperature as it is the origin of cosmological redshift. The energy of massive free particles in the universe scales differently based on their momentum (and thus temperature). When hot and relativistic, particle energy scales with inverse scale factors like radiation. However as particles transition to non-relativistic momenta, their energies scale with the inverse square of the scale factor like magnetic flux.
\begin{alignat}{1}
    \label{EScale} E(t) = E(t_{0})\frac{a(t_{0})}{a(t)}\xrightarrow{\mathrm{NR}}\  E(t_{0})\frac{a(t_{0})^{2}}{a(t)^{2}}\,.
\end{alignat}
This occurs because of the functional dependence of energy on momentum in the relativistic versus non-relativistic cases.

The argument in the Boltzmann statistical factor is given by
\begin{alignat}{1}
    \label{Boltz} X_{n}^{s}\equiv\frac{E_{n}^{s}}{T}\,.
\end{alignat}
We can explore this relationship for the magnetized system explicitly by writing out \req{Boltz} using the KGP eigen-energies as
\begin{alignat}{1}
    \label{XExplicit} X_{n}^{s} = \sqrt{\frac{m_{e}^{2}}{T^{2}}+\frac{p_{z}^{2}}{T^{2}}+\frac{2eB}{T^{2}}\left(n+\frac{1}{2}-\frac{gs}{2}\right)}\,,
\end{alignat}
where we now introduce the expansion scale factor via \req{BScale} - \req{PScale}. The Boltzmann factor can then be written as
\begin{alignat}{1}
    \label{XScale} X_{n}^{s}[a(t)] = \sqrt{\frac{m_{e}^{2}}{T^{2}(t_{0})}\frac{a(t)^{2}}{a(t_{0})^{2}}+\frac{p_{z}^{2}(t_{0})}{T^{2}(t_{0})}+\frac{2eB(t_{0})}{T^{2}(t_{0})}\left(n+\frac{1}{2}-\frac{gs}{2}\right)}\,.
\end{alignat}
This reveals that only the mass contribution is dynamic over cosmological time. For any given eigen-state, the mass term increases driving the state into the non-relativistic limit while the momenta and magnetic contributions are frozen by initial conditions. 

As a point of comparison, the Boltzmann factor for the DP eigen-energies becomes
\begin{alignat}{1}
    \label{XDP} X_{n}^{s}\vert_{DP} = \sqrt{\left(\sqrt{\frac{m_{e}^{2}}{T^{2}}+\frac{2eB}{T^{2}}\left(n+\frac{1}{2}-s\right)}-\frac{eB}{2mT}(g-2)s\right)^{2}+\frac{p_{z}^{2}}{T^{2}}}\,,
\end{alignat}
which over cosmological time under expansion scales as
\begin{multline}
    \label{XScaleDP} X_{n}^{s}[a(t)]\vert_{DP} =\\ \sqrt{\left(\sqrt{\frac{m_{e}^{2}}{T^{2}(t_{0})}\frac{a(t)^{2}}{a(t_{0})^{2}}+\frac{2eB(t_{0})}{T^{2}(t_{0})}\left(n+\frac{1}{2}-s\right)}-\frac{eB(t_{0})}{2mT(t_{0})}\frac{a(t_{0})}{a(t)}(g-2)s\right)^{2}+\frac{p_{z}^{2}(t_{0})}{T^{2}(t_{0})}}\,.
\end{multline}
While the above expression is rather complicated, we note that the KGP~\req{XScale} and DP~\req{XDP} Boltzmann factors both reduce to the Sch{\"o}dinger-Pauli limit as $a(t)\rightarrow\infty$ thereby demonstrating that the total magnetic moment is protected under the adiabatic expansion of the universe.

Higher order non-minimal magnetic contributions which can be introduced to the eigen-energies like $\approx\mu_{B}^{2}B^{2}/T^{2}$ are then suppressed over cosmological time driving the system into minimal electromagnetic couplings with the exception of the anomalous magnetic moment. It is interesting to note that cosmological expansion serves to \lq smooth out\rq\ the characteristics of more complex BSM electrodynamics erasing them from a statistical perspective in favor of the minimal or minimal-like dynamics.