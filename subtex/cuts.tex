Before we handle an ensemble system, we will look at the magnetic moment $\boldsymbol{\mu}$ of a single-particle quantum system. To avoid confusion, the permeability will always be denoted either by a subscript for the vacuum or medium to always differentiate from magnetic moment. If we apply an external field with the local constant value of $B$ which is primarily responsible for the particle's response, the magnetic moment can be evaluated from the matrix element
\begin{alignat}{1}
  \label{CHIeq03} \left|\boldsymbol{\mu}_{n}\right|=-\left\langle n\left|\partial\hat{\mathcal{H}}/\partial B\right|n\right\rangle=-\frac{\partial E_{n}}{\partial B}\,,
\end{alignat}
where $\hat{\mathcal{H}}$ is the Hamiltonian of the system. It is valuable to point out that the individual dipole's response within a medium is only uniquely dependent on the external field $\textbf{H}$ in the case of weak medium magnetization. However, if the bulk magnetization is may easily be large and thus influential to each individual dipole. Physically each dipole is sensitive to the total magnetic flux $\textbf{B}$ which includes a mixture of external field and bulk magnetization $\textbf{M}$ from its neighbors. The magnetization density of the quantum system is then
\begin{alignat}{1}
  \label{CHIeq04} M_{n}(B)=\frac{1}{V}\left|\boldsymbol{\mu}_{n}\right|\,.
\end{alignat}
If we couple the system to a thermal reservoir of temperature $T$, the averaged magnetization at thermal and chemical equilibrium is
\begin{alignat}{1}
  \label{CHIeq05} M(B,T,\eta)=\frac{\sum_{n}M_{n}e^{-\beta (E_{n}-\eta N)}}{\sum_{n}e^{-\beta (E_{n}-\eta N)}}\,,
\end{alignat}
where $\beta=1/k_{B}T$, $k_{B}$ is the Boltzmann constant and $\eta$ is the chemical potential. We then introduce the grand potential $\Phi$ defined by
\begin{alignat}{1}
  \label{CHIeq06} \Phi=-\frac{1}{\beta}\ln\left({\sum_{n}e^{-\beta (E_{n}-\eta N)}}\right)=-\frac{1}{\beta}\ln\left(\mathcal{Z}\right)\,,
\end{alignat}
where $\mathcal{Z}$ is the grand partition function. This allows us to rewrite eq.~\eqref{CHIeq05} as
\begin{alignat}{1}
  \label{CHIeq07} M(B,T,\eta)=-\frac{1}{V}\frac{\partial \Phi}{\partial B}\,.
\end{alignat}
Combining eq.~\eqref{CHIeq06} and eq.~\eqref{CHIeq07} in the grand ensemble yields
\begin{alignat}{1}
  \label{Mag} M=\frac{1}{\beta V}\frac{\partial}{\partial B}\ln\left(\mathcal{Z}\right)\,.
\end{alignat}
The magnetic susceptibility is related to the magnetization via
\begin{alignat}{1}
  \label{CHIeq09} \chi=\mu_{vac.}\frac{\partial M}{\partial B}\,.
\end{alignat}
If a given thermodynamic system is well described by a partition function, we can evaluate the susceptibility using eq.~\eqref{CHIeq09}.








We note here one important difference between KGP and DP eigen-energies in the context of cosmology: The anomalous magnetic moment portion of the DP statistics is suppressed by $1/a(t)$ over cosmological time while the AMM contribution is preserved in the KGP model. That the universe's expansion makes a distinction between $g=2$ magnetic moment and AMM for DP fermions appears as a rather artificial and nonphysical trait. While the suppression of AMM may often be small for particles such as electrons, this suppression is non-trivial for particles with large AMM values such as the proton. That cosmological redshift would push DP protons to be described by $g=2$ eign-energies in the non-relativistic limit counts as a malaise for the model and further strengthens our thinking that the KGP model is more appropriate for cosmological studies. Motivated by \req{XScale}, we can introduce a dimensionless cosmic magnetic scale which is frozen in the homogeneous case as
\begin{alignat}{1}
    \label{Bo} b_{0}\equiv\frac{eB}{T^{2}}=\frac{eB\hbar c^{2}}{(k_{B}T)^{2}}\ \mathrm{(S.I)}\,,
\end{alignat}
where we've included the expression explicitly in full SI units. We can estimate the value of $b_{0}$ from the bounds on the extra-galactic magnetic field strength and the temperature of the universe today.  If the origin of deep space extra-galactic magnetic fields are relic fields from the early universe, which today are expected to exist between $5\times10^{-12}\ \mathrm{T}>B_{relic}>10^{-20}\ \mathrm{T}$, then at temperature $T=2.7\ \mathrm{K}$, the value of the cosmic magnetic scale is between
\begin{alignat}{1}
    \label{BoScale} 5.5\times10^{-5}>b_{0}>1.1\times10^{-11}\,.
\end{alignat}
This should remain constant in the universe at-large up to the last epoch the universe was sufficiently magnetized to disturb this value. As the electron-proton plasma which generated the CMB was relatively dilute over its duration, it was unlikely sufficiently magnetized to significantly alter this value over extra-galactic scales. Rather, the best candidate plasma to have been sufficiently magnetized and dense to have set the relic field magnetic scale would have been the electron-positron plasma which existed during the duration of Big Bang Nucleosynthesis (BBN) and beforehand.

As $b_0$ is a constant of expansion, assuming the electron-proton plasma between the CMB and electron-positron annihilation did not greatly disturbed it, we can calculate the remnant values at the temperature $T=50\ \mathrm{keV}$ with the expression
\begin{align}
  \label{BBNFields} B(T)=\frac{b_{0}}{e}T^{2}\,,
\end{align}
yielding a range of field strengths
\begin{align}
  \label{BBNRange} 2.3\times10^{5}\ \mathrm{T}>B(T=50\ \mathrm{keV})>4.6\times10^{-4}\ \mathrm{T}\,,
\end{align}
during which the electron-positron plasma in the universe had a number density comparable to that of the Solar core with $n_{e}=4.49\times10^{24}\ \mathrm{cm}^{-3}$ at $r=0.01R_{\odot}$.

While we consider the $g$-factor to be the immutable description of magnetic moment, it is useful in the case of Landau levels to consider the anomaly $a$ as from the energies given in eq.~\eqref{LANeq02}, $g$ always is linearly added or subtracted from the Landau quantum number which are integers.

Degeneracy is restored for values of the anomalous parameter given in eq.~\eqref{LANeq04}. While generating a large number of eccentric states from large anomalous moment may be of theoretical interest, it is of practical interest to consider particles like the proton or electron where only the ground state is uniquely disturbed. Due to the properties of logs, the overall partition function will be sum of the partition function of each species, which allows us to consider each species separately.

The magnetization of this term, defined by \req{Mag}, is found to be
\begin{alignat}{1}
    \label{FreelikeMag} M_{F}^{s,\sigma}=\frac{T}{V}\frac{\partial m_{s}}{\partial B}\frac{\partial}{\partial m_{s}}\ln(\mathcal{Z}_{F}^{\sigma})\,\,,
\end{alignat}
with the total magnetization given by the sum over the four species
\begin{alignat}{1}
    \label{TotalFreeMag} M_{F}=\sum_{s,\sigma}M_{F}^{s,\sigma}\,.
\end{alignat}
The term by term magnetization evaluates as
\begin{multline}
  \label{MagExplicit} M_{F}^{s,\sigma} = \frac{e(1-gs)}{2m_{s}}\frac{5}{m_{s}}\ln\left(\mathcal{Z}^{\sigma}_{F}\right)|_{m_{s}}\\
  -\frac{e(1-gs)}{2m_{s}}\frac{1}{3}\frac{L^{3}}{(2\pi)^{2}}m_{s}^{3}\left(\frac{m_{s}}{T}\right)^{2}
  \int_{-\infty}^{+\infty}dt\left[z_{\sigma}^{-1}e^{a_{s}\cosh(t)}\cosh(t)\sinh^{4}(t)F^{2}\left[X_{s},\sigma\right]\right]\,,
\end{multline}












%%%%%%%%%%%%%%%%%%%%%%%%%%%%%%%%%%%%%%%
\section{Connecting quantum and classical theory}
\label{sec:ehrenfest}
%%%%%%%%%%%%%%%%%%%%%%%%%%%%%%%%%%%%%%%
\noindent The relativistic precession of a spin-1/2 particle can be extracted from the Dirac equation. The Dirac equation with electromagnetic interaction is given by
\begin{alignat}{1}
  \label{DIRAC01} i\hbar\frac{\partial}{\partial t}\psi=\left(\boldsymbol{\alpha}\cdot\left(\mathbf{p}c-e\mathbf{A}\right)+eA_{0}+\gamma^{0}mc^{2}\right)\psi=\hat{H}_{D}\psi\,,
\end{alignat}

In the Heisenberg representation, operators obeys the following equation of motion
\begin{alignat}{1}
  \label{m06}		&i\hbar\frac{\mathrm{d}\hat{O}}{\mathrm{d}t}-i\hbar\frac{\partial\hat{O}}{\partial t}=[\hat{O},\hat{H}]
\end{alignat}
and the expectation value is of course
\begin{alignat}{1}
  \label{m07}		&\langle i\hbar\frac{\mathrm{d}\hat{O}}{\mathrm{d}t}\rangle-\langle i\hbar\frac{\partial\hat{O}}{\partial t}\rangle=\langle[\hat{O},\hat{H}]\rangle
\end{alignat}

The operator spin precession is
\begin{alignat}{1}
  \label{DIRAC04} \frac{\mathrm{d}\mathbf{S}_{R}}{\mathrm{d}t}=\frac{\mathrm{d}}{\mathrm{d}t}\left(\frac{\hbar}{2}\boldsymbol{\Sigma}\right)=-\boldsymbol{\alpha}\times\left(\mathbf{p}c-e\mathbf{A}\right)\,,
\end{alignat}
where $\mathbf{S}_{R}$ is the relativistic spin operator. While this expression is fully relativistic, it is not easily interpreted due to the off-diagonal nature of the $\boldsymbol{\alpha}$ matrices which mix particle and antiparticle components. A similar situation arises for the `velocity operator' which leads to the phenomenon of \emph{Zitterbewegung}. Eq.~\eqref{DIRAC04} reveals that spin is not a constant of motion of the Dirac equation. The orbital angular momentum $\mathbf{L}$ is also not a constant of motion, but the sum of the two or the total angular momentum $\mathbf{J}$ is.

The expression for the Lorentz force derivable from the Schr{\"o}dinger-Pauli equation is given by
\begin{alignat}{1}
  \label{m31}		&\frac{\mathrm{d}\pi_{k}}{\mathrm{d}t}=eE_{k}+\frac{e}{2mc}(\vec{\pi}\times\vec{B}-\vec{B}\times\vec{\pi})_{k}+\frac{e\hbar}{2mc}\vec{\sigma}\cdot\nabla_{k}\vec{B}
\end{alignat}
which contains the Stern-Gerlach force and correct magnetic moment with $g\!=\!2$.

%%%%%%%%%%%%%%%%%%%%%%%%%%%%%%%%%%%%%%%
\subsubsection{Foldy-Wouthuysen transformation}
%%%%%%%%%%%%%%%%%%%%%%%%%%%%%%%%%%%%%%%
To connect the relativistic quantum notion of spin to our non-relativistic classical intuition we need to disentangle the components of the Dirac equation. To do this we make use of the Foldy-Wouthuysen (FW) transformation which diagonalizes odd operators via a unitary transformation
\begin{alignat}{1}
  \label{FW01} \psi\rightarrow\psi'=e^{i\mathcal{S}}\psi,\indent\hat{H}\rightarrow\hat{H}'=e^{i\mathcal{S}}\hat{H}e^{-i\mathcal{S}}-ie^{i\mathcal{S}}\frac{\partial e^{-i\mathcal{S}}}{\partial t}\,,
\end{alignat}
where $\mathcal{S}$ is Hermitian. The resulting Hamiltonian $\hat{H}'$ then produces two uncoupled equations of two-component spinors. For free particles the transformed Hamiltonian has an exact closed form, but for charged particles in weak fields the Hamiltonian will be an infinite series in powers of inverse mass $1/m$. Our interest lies in determining the first nonlinear terms in the spin precession of order $|\mathbf{s}|^{2}$ as these are the first terms to be sensitive to field in-homogeneity. Because of the intimate relationship between spin and magnetic moment, the expressions we seek will be up to order $\mu^{2}\sim e^{2}/m^{2}$ and thus we only require the FW transformed Hamiltonian up to $1/m^{2}$.

The generated Hamiltonian is therefore
\begin{alignat}{1}
  \notag\hat{H}'&=\gamma^{0}\left\{mc^{2}+\frac{1}{2m}\left(\mathbf{p}-\frac{e}{c}\mathbf{A}\right)^{2}-eA_{0}-\frac{e\hbar}{2mc}\boldsymbol{\sigma}\cdot\mathbf{B}\right.\\
  \notag&+\frac{e\hbar}{8m^{2}c^{2}}\boldsymbol{\sigma}\cdot\left(\left(\mathbf{p}-\frac{e}{c}\mathbf{A}\right)\times\mathbf{E}-\mathbf{E}\times\left(\mathbf{p}-\frac{e}{c}\mathbf{A}\right)\right)\\
  \label{FW02}&-\left.\frac{e\hbar^{2}}{8m^{2}c^{2}}\boldsymbol{\nabla}\cdot\mathbf{E}+\ldots\right\}\,.
\end{alignat}
The $\boldsymbol{\sigma}$ operators are simply the Pauli matrices. The first four terms represent the rest mass-energy of the particle and the Schr\"{o}dinger-Pauli (SP) Hamiltonian, the fifth term is the spin-orbit coupling, and the last is the Darwin term. In determining the precession, it is the spin Hamiltonian
\begin{alignat}{1}
  \notag\hat{H}'_{spin}=&-\frac{e\hbar}{2mc}\boldsymbol{\sigma}\cdot\bigg(\mathbf{B}\\
  \label{FW03}&-\frac{1}{4mc}\left(\left(\mathbf{p}-\frac{e}{c}\mathbf{A}\right)\times\mathbf{E}-\mathbf{E}\times\left(\mathbf{p}-\frac{e}{c}\mathbf{A}\right)\right)\bigg),
\end{alignat}
which will be relevant.

%%%%%%%%%%%%%%%%%%%%%%%%%%%%%%%%%%%%%%%
\subsection{Second order spin effects in the Stern-Garlach force}
\label{sec:spinspin}
%%%%%%%%%%%%%%%%%%%%%%%%%%%%%%%%%%%%%%%
By substituting eq.~\eqref{FW03} into eq.~\eqref{DIRAC03}, the non-relativistic spin precession (up to order $1/m^{2}$) is found to be
\begin{alignat}{1}
  \notag\frac{\mathrm{d}\mathbf{S}_{NR}}{\mathrm{d}t}=\frac{\mathrm{d}}{\mathrm{d}t}\left(\frac{\hbar}{2}\boldsymbol{\sigma}\right)&=\frac{e\hbar}{2mc}\boldsymbol{\sigma}\times\mathbf{B}\\
  \notag&-\frac{e\hbar}{4m^{2}c^{2}}\boldsymbol{\sigma}\times\left(\mathbf{E}\times\left(\mathbf{p}-\frac{e}{c}\mathbf{A}\right)\right)\\
  \label{SPIN01}&-\frac{e\hbar^{2}}{16m^{2}c^{2}}\left[\boldsymbol{\sigma},\boldsymbol{\sigma}\cdot\left(\boldsymbol{\nabla}\times\mathbf{E}\right)\right]\,,
\end{alignat}
where $\mathbf{S}_{NR}$ is the non-relativistic spin operator. The first term is the expected Larmor precession, and the second is precession generated by spin-orbit coupling. The last term is second order in spin and sensitive to the curl of electric fields. This term cannot be easily categorized as a simple `quantum correction' as both orders of $\hbar$ are consumed by the definition of the spin operator and survive the $\hbar\rightarrow 0$ limit. This term in fact is more analogous to the $p^{4}$ correction to a particle's energy due to relativity.

As spin dynamics in quantum mechanics exhibits more complicated behavior at higher orders, due to special relativity, we expect similar corrections to appear in the classical theory of spin precession in the same manner that energy corrections appear.










\subsubsection{Neutral-Coulomb Problem}\label{ajsss:neutral}
\noindent An important consequence of the magnetic dipole moment is that it allows otherwise neutral particles to participate electromagnetically with charged systems. While this is most obviously important for the neutron which is easily attracted to matter through dispersion forces, this has also consequences for the neutrino which is an ellusive particle due to the small coupling of the weak interaction which it is sensitve to. The theoretical magnetic dipole for the neutrino also allows for an avenue to more explore the pure electromagnetic behavior of dipoles than neutrons. {\color{red}Add statements on the neutron dipole size.} Neutrons interact with matter through the residual strong force fairly easily which complicates the study of electromagnetism under strong field systems alone. The neutrino on the otherhand only has weak coupling which is less affecting making the neutrino a more ideal candidate to explore electromagnetism in strong field systems.

{\color{red}Add statements discussing the various estimates for the neutrino magnetic moment size.}

The most straight-forward manner to explore the neutral-Coulomb system is to write down the effective potential and check the zeros of the potential for the possibility of bound-states. While general wisdom suggests that neutral particles cannot have bound states with charged particles, there are cases where special unique singular bound states are allowed and additionally magnetic dipole moments can lead to mass resonances as features of scattering due to the disruptive nature of magnetic dipole moments on angular momentum.














%%%%%%%%%%%%%%%%%%%%%%%%%%%%%%%%%%%%%%%
\subsection{Flavor, mass and magnetic eigenstates}\label{sec:mix}
%%%%%%%%%%%%%%%%%%%%%%%%%%%%%%%%%%%%%%%
\noindent As neutrinos have masses, there is no guarantee that their $SU(2)_{L}$ flavor eigenstates will be simultaneously their propagating mass eigenstates. This misalignment between the two representations can then written as rotation of the neutrino flavor 3-vector where $N=3$ is the number of generations. The unitary mixing matrix $\bb{V}$ allows for the change of basis between mass and flavor eigenstates via
\begin{alignat}{1}
	\label{basis:1} \bb{\nu_{f}}=\bb{V}\bb{\nu_{m}}\,\rightarrow\indent
	\begin{pmatrix}
		\nu_{e}\\
		\nu_{\mu}\\
		\nu_{\tau}
	\end{pmatrix}=
	\begin{pmatrix}
		V_{e1} & V_{e2} & V_{e3}\\
		V_{\mu1} & V_{\mu2} & V_{\mu3}\\
		V_{\tau1} & V_{\tau2} & V_{\tau3}
	\end{pmatrix}
	\begin{pmatrix}
		\nu_{1}\\
		\nu_{2}\\
		\nu_{3}
	\end{pmatrix}\,,
\end{alignat}
where $\bb{\nu_{f}}$ is the neutrino state vector written in the flavor basis while $\bb{\nu_{m}}$ is written in the mass basis. Boldface type will be used for matrices and vectors. Bars atop vectors represent the Dirac adjoint in the usual manner. The mixing matrix's form then depends on the Dirac-like or Majorana-like nature of the neutrinos
\begin{align}
	\label{phases:1} \bb{V} = \bb{U}\bb{P}\,,\\
	\label{phases:2} \bb{P}_\mathrm{Dirac} = \mathbbm{1}\,,\\
	\label{phases:3} \bb{P}_\mathrm{Maj.} = \mathrm{diag}(e^{i\rho},e^{i\sigma},1)\,.
\end{align}
Majorana neutrinos allow up to two additional complex phases $\rho$ and $\sigma$ which participate in CPV. In the standard parameterization~\citep{Schwartz:2014sze}, the rotation matrix $\bb{U}$ can be expressed as
\begin{alignat}{1}
	\label{rotation:1} \bb{U} =
	  \begin{pmatrix}
		  c_{12}c_{13} & s_{12}c_{13} & s_{13}e^{-i\delta}\\
		  -s_{12}c_{23} - c_{12}s_{13}s_{23}e^{i\delta} & c_{12}c_{23} - s_{12}s_{13}s_{23}e^{i\delta} & c_{13}s_{23}\\
		  s_{12}s_{23} - c_{12}s_{13}c_{23}e^{i\delta}& -c_{12}s_{23} - s_{12}s_{13}c_{23}e^{i\delta} & c_{13}c_{23}
	  \end{pmatrix}\,,
\end{alignat}
where $c_{ij} = \rm{cos}(\theta_{ij})$ and $s_{ij} = \rm{sin}(\theta_{ij})$. In this convention, the three mixing angles $(\theta_{12}, \theta_{13}, \theta_{23})$, are understood to be the Euler angles for generalized rotations. There are many possible parametrizations for the mixing matrix and without a working model of the underlying physics, they represent generic observables which are otherwise not predicted. Another relevant choice is the Wolfenstein parameterization~\citep{wolfenstein1983parametrization}, but as neutrinos mixing angles are rather large unlike the parameters for the CKM matrix in the quark sector, we will not use it here. The Majorana mass Lagrangian in the flavor basis can then be written as
\begin{alignat}{1}
	\label{mass:1} -\mathcal{L}_{\rm{mass}}^{\rm{Maj.}}&=\frac{1}{2}\bb{\bar{\nu}_{f}^{L}}\bb{M}_{\nu}\left(\bb{\nu_{f}^{L}}\right)^{c}+\rm{h.c}\,,
\end{alignat}
where $\bb{\nu^{L}}$ refers to left-handed chiral states which can be obtained using projection operators and $\gamma^{5}$. The superscript $\bb{\nu}^{c}$ refers to the charge conjugated state where $\bb{\nu}^{c} = \hat{C}(\bb{\bar{\nu}})^\mathrm{T}$ is the charge conjugate of the neutrino field. The operator $\hat{C} = i\gamma^{2}\gamma^{0}$ is the charge conjugation operator which can be written as a $4\times4$ matrix for a given representation as each flavor is in this formulation a four-component spinor.

The source of CPV in the neutrino sector is ultimately attributable to the fundamental mismatch between the mass-matrices of the charged leptonic flavors $(e,\mu,\tau)$ and the neutrino flavors $(\nu_{e},\nu_{\mu},\nu_{\tau})$. The situation is analogous to the quark sector, where instead the relation is between the upper $(u,c,t)$ and lower quark $(d,s,b)$ flavors. This means that the mass matrix for charged leptons does not commute with the mass matrix of the neutral leptons and cannot be simultaneously diagonalized except for special cases or degeneracy among the mass eigenstates. We can characterize CPV by introducing the mixing matrices $\bb{V}$ which diagonalize the individual mass-matrices $\bb{M}$ as follows
\begin{alignat}{1}
	\label{diag:1} \bb{V}\bb{M}_{\nu}\bb{M}_{\nu}^{\dagger}\bb{V}^{\dagger} = \bb{D}_{\nu}^{2} = \rm{diag}(m_{1}^{2},m_{2}^{2},m_{3}^{2})\,,\\
    \label{diag:2} \bb{M}_{\ell} \equiv \bb{D}_{\ell} = \rm{diag}(m_{e},m_{\mu},m_{\tau})\,,
\end{alignat}
where the subscript $\nu$ refers to neutrino states while $\ell$ refers to charged lepton states. We have specifically defined the charged leptons flavor states as being simultaneously mass eigenstates without a loss of generality. We will not consider oscillation among the charged leptons, though that may be an avenue of further study~\citep{akhmedov2007charged}. We note that being unitary, the matrix $\bb{V}$ can diagonalize the Hermitian square of the mass matrices as well as the mass matrices themselves.