%%%%%%%%%%%%%%%%%%%%%%%%%%%%%%%%%%%%%%%
\chapter{Future research efforts}
\label{chap:future}
%%%%%%%%%%%%%%%%%%%%%%%%%%%%%%%%%%%%%%%

This chapter contains unpublished work and is exploratory; though we hope to complete this research in the near future. \rsec{sec:quarks} extends the KGP wave equation to also include chromomagnetic dipole moments necessary for quarks in quantum chromodynamics (QCD). We show explicitly that the dipoles are linear in field tensors and do not mix spin degrees of freedom.

In \rsec{sec:nucp}, we explore CP violation in the neutrino sector in terms of the Jarlskog invariant and modifications to the mass matrix. The CP symmetry of the electric dipole is discussed. This section follows the conventions found in \rchap{chap:neutrino}.

The final \rsec{sec:kk} is a short description of a personal passion project to integrate spin dynamics into the five dimensional theory of Kaluza-Klein which unifies electromagnetism and gravitation classically. While I have written much on the topic, it remains unfinished and will only occupy a small section of this dissertation.

%%%%%%%%%%%%%%%%%%%%%%%%%%%%%%%%%%%%%%%
\section{Klein-Gordon-Pauli extensions to non-Abelian fields}
\label{sec:quarks}
%%%%%%%%%%%%%%%%%%%%%%%%%%%%%%%%%%%%%%%

%I'm leaving this bit in as an easteregg since I wanted to do g!=2 stuff with spin-1 fields but never got far into it while writing this. When I return to in the future, I'll come back here resurrect this bit.
%While most attention in particle physics focuses on the dipole moments of fermions, it is well known that bosons with spin can also carry dipole moments. The most pertinent example is that of the W-boson which after electroweak symmetry breaking has a magnetic dipole term which is analogous to the Pauli term in KGP or DP for fermions. While a wide variety of spin-1 formulations exist, such as the Duffin-Kemmer formulation \ar, in the following discussion we will write in the Proca formulation as it is most similar to how electromagnetic fields, which are also spin-1 fields, are written. For a complex spin-1 field, the Proca action is given by
%\begin{alignat}{1}
%	\label{eq:proca:01a} \mathcal{L}_{\mathrm{Proca}} = -\frac{1}{2}\mathcal{F}_{\mu\nu}^{*}\mathcal{F}^{\mu\nu}+\frac{m^{2}c^{2}}{\hbar^{2}}B_{\mu}^{*}B^{\mu}\,,\\
%	\label{eq:proca:01b} \mathcal{F}^{\mu\nu}=\partial^{\mu}B^{\nu}-\partial^{\nu}B^{\mu}\,.
%\end{alignat}
%The Euler-Lagrangian equations of motion are then
%\begin{alignat}{1}
%	\label{eq:proca:02a} \partial_{\mu}\mathcal{F}^{\mu\nu}+\frac{m^{2}c^{2}}{\hbar^{2}}B^{\nu}=0\,,
%\end{alignat}
%which if the Lorenz gauge $\partial\cdot B=0$ is taken simplifies to a Klein-Gordon style second-order wave equation. If we gauge the fields giving the spin-1 field an electric charge, we can produce a theory which naturally generates a magnetic moment.

The KGP approach to wave equations is robust and is useful not only for electromagnetic interactions: This is a first look at adding the strong interaction into KGP inspired in part by~\cite{Labun:2012ra}. Quarks participate in the strong color interaction of quantum chromodynamics (QCD); therefore they should present a $g$-factor for both their electromagnetic dipole $g_\mathrm{EM}$ as well as their QCD chromomagnetic dipole $g_\mathrm{QCD}$.

Since color charges follows a more complex $SU(3)$ group structure unlike the more straight forward $U(1)$ of electromagnetism, the ``color magnetism'' of QCD requires more than just the analogous Pauli term to describe color dipole moments owing due to the fact that QCD has non-Abelian (non-commuting) gluon gauge fields $\mathcal{A}^{\alpha}$.

The quarks (like all particles) should obey the quantum mechanical analogue of the energy-momentum relation seen in \req{eq:spin:03} with the only theoretical difference being a modified covariant derivative. The covariant derivative, written in terms of kinetic momentum, should appear as
\begin{alignat}{1}
    \label{eq:spin:08}
    \mathrm{EM+QCD}:\qquad i\hbar\widetilde\nabla=\pi^{\alpha}=p^{\alpha}-q_\mathrm{EM}A^{\alpha}-q_\mathrm{QCD}\mathcal{A}^{\alpha}\,,
\end{alignat}
where $q_\mathrm{EM}$ is the electric charge of the quarks $q_\mathrm{EM}/e\in\pm1/3,\pm2/3$ and $q_\mathrm{QCD}$ is the color charge coupling strength. In many texts the symbol $g_{s}$ is used for the color coupling strength, but we circumvent that notation using $q_\mathrm{QCD}$ to avoid confusion with $g$-factor.

We follow the conventions in~\cite{greiner2006qcd}. The eight $3\times3$ Gell-Mann matrices $\lambda^{a}$ are embedded into each independent field as
\begin{alignat}{1}
	\label{eq:spin:09} \mathcal{A}^{\alpha}\equiv\frac{1}{2}\lambda^{a}\mathcal{A}^{\alpha}_{a}\,,\qquad a\in1\ldots8\,,\qquad
    [\lambda^{a},\lambda^{b}]=\frac{i}{2}f^{abc}\lambda^{c}\,,
\end{alignat}
where $\mathcal{A}_{a}^{\alpha}$ are the individual fields for each gluon species in a given representation and $f^{abc}$ is the $SU(3)$ antisymmetric structure function. The non-commuting behavior of the Gell-Mann matrices captures the non-Abelian structure of the gauge fields. The gluon field strength tensor $\mathcal{G}^{\alpha\beta}$ is then
\begin{gather}
	\label{eq:spin:10a}
    \mathcal{G}^{\alpha\beta} = \partial^{\alpha}\mathcal{A}^{\beta} -\partial^{\beta}\mathcal{A}^{\alpha} + \frac{i}{\hbar}q_\mathrm{QCD}\left[\mathcal{A}^{\alpha},\mathcal{A}^{\beta}\right]\,,\\
	\label{eq:spin:10b} \left[\mathcal{A}^{\alpha},\mathcal{A}^{\beta}\right] =
    \frac{1}{4}\mathcal{A}^{\alpha}_{a}\mathcal{A}^{\beta}_{b}\left[\lambda^{a},\lambda^{b}\right] =
    \frac{i}{8}\mathcal{A}^{\alpha}_{a}\mathcal{A}^{\beta}_{b}f^{abc}\lambda_{c}\,.
\end{gather}

Following the same procedure in \rsec{sec:unique}, we can generalize the energy-momentum relation and obtain the EM+QCD variant of the KGP equation for quarks. We find that the resulting quark-KGP equation is
\begin{align}
    \label{eq:spin:11a}
    \gamma_{\alpha}\gamma_{\beta}\pi^{\alpha}\pi^{\beta} = \eta_{\alpha\beta}\pi^{\alpha}\pi^{\beta} - 
    \frac{q_\mathrm{EM}\hbar}{2}\sigma_{\alpha\beta}F^{\alpha\beta} - 
    \frac{q_\mathrm{QCD}\hbar}{2}\sigma_{\alpha\beta}\mathcal{G}^{\alpha\beta}\,,\\
	\label{eq:spin:11b} \left(\eta_{\alpha\beta}\pi^{\alpha}\pi^{\beta} - 
    \frac{q_\mathrm{EM}\hbar}{2}\sigma_{\alpha\beta}F^{\alpha\beta} - 
    \frac{q_\mathrm{QCD}\hbar}{2}\sigma_{\alpha\beta}\mathcal{G}^{\alpha\beta}\right)\Psi=m_{q}^{2}c^{2}\Psi_{q}\,,
\end{align}
which mirrors the electromagnetic case except for the extension of a chromomagnetic Pauli term. We note that $m_{q}$ is the quark mass and that the field $\Psi_{q}$ is a color triplet of spinors for: red, green, blue:
\begin{align}
    \Psi_{q}=
    \begin{pmatrix}
        \Psi_{r}\\
        \Psi_{g}\\
        \Psi_{b}
    \end{pmatrix}\,.
\end{align}
As only the non-commuting portion of \req{eq:spin:11b} (written explicitly in \req{eq:spin:10b}) is off-diagonal in color, this means that the additional non-commuting chromomagnetic term acts as a transition matrix between different quark colors.

This method (based on the commutator of the kinetic momentum) suggests that the color and electromagnetic $g$-factors both have a natural value of $g_\mathrm{EM}\!=\!g_\mathrm{QCD}\!=\!2$. We can generalize \req{eq:spin:11a} to allow for arbitrary EM and QCD dipole moments $g_\mathrm{EM}$ and $g_\mathrm{QCD}$ respectively as
\begin{align}
	\label{eq:spin:12}
    \boxed{\left(\eta_{\alpha\beta}\pi^{\alpha}\pi^{\beta}-\frac{g_\mathrm{EM}}{2}\frac{q_\mathrm{EM}\hbar}{2}\sigma_{\alpha\beta}F^{\alpha\beta}-\frac{g_\mathrm{QCD}}{2}\frac{q_\mathrm{QCD}\hbar}{2}\sigma_{\alpha\beta}\mathcal{G}^{\alpha\beta}\right)\Psi=m_{q}^{2}c^{2}\Psi}\,.
\end{align}


While in electromagnetism, DP and KGP approaches only differ in the presence of strong EM fields and are otherwise identical in the weak field limit, this cannot be equally said in QCD. The perturbative limit which justifies the DP approach for leptons in QED is possible due to the small value of the fine structure constant which is not true in QCD. Only for high momentum interactions (such as those present in quark-gluon-plasma (QGP) or in energetic collisions) is the perturbative approach applicable~\citep{Choudhury:2014lna}. We emphasize however that the DP approach is only valid where the dipole moment is obtainable via perturbative expansion which may not hold if the $g$-factor results from non-perturbative physics.

The dipole characteristics (both electromagnetic and chromomagnetic) of the top-quark is of particular interest~\citep{Labun:2012fg,Vryonidou:2018eyv} because of top-quark's strong coupling to the Higgs and potential BSM physics. We note that current studies focus on a DP approach to chromomagnetism~\citep{Zhang:2010dr,Zhang:2012muc,BuarqueFranzosi:2015jrv}. There is also the added complexity of both the chromomagnetic and magnetic $g$-factors differing from the natural value independently of one another $g_\mathrm{EM}\neq g_\mathrm{QCD}\neq2$. Further study of the KGP approach to chromomagnetism should be pursued. To our knowledge, there is no formulation of \req{eq:spin:12} as an effective field theory in the manner of~\cite{Fleming:2000ib,Bauer:2000yr} for quarks or for electromagnetically charged fermions as was discussed in \rsec{sec:lagrangian}.


%%%%%%%%%%%%%%%%%%%%%%%%%%%%%%%%%%%%%%%
\section{Electromagnetic forced neutrino CP-violation}
\label{sec:nucp}
%%%%%%%%%%%%%%%%%%%%%%%%%%%%%%%%%%%%%%%
Electromagnetic processes in the neutrino sector may yield measurable effects in two aspects of neutrino physics: 
\begin{itemize}
    \item[(a)] Neutrino oscillation which is evidence of the non-zero mass eigenstates
    \item[(b)] Charge-parity (CP) violation in the neutrino sector which occurs due to the presence of at least three generations of neutrinos or additional CP violating interactions
\end{itemize}

Our motivation is to explore the effect of strong electromagnetic fields on neutrino CP violation by analysis of the electromagnetic dipole interaction and determining its influence on the Jarlskog invariant $J$~\citep{Jarlskog:1985ht,Jarlskog:1985cw,Jarlskog:2004be} which controls the size of CP violation.

One important aspect of neutrino physics is the size of the CP violation~\citep{Xing:2000ik,giunti2007fundamentals,Huber:2022lpm} which can yield insights not only in fundamental physics but cosmology as well. One of the goals of modern neutrino experiments such as DUNE~\cite{DUNE:2020jqi} is to better characterize such effects in astrophysical contexts such as supernova~\citep{DUNE:2020zfm,SajjadAthar:2021prg}, solar neutrinos~\citep{Akhmedov:2022txm} and magnetars~\citep{Lichkunov:2020zzx}. Additionally, there may be a connection~\citep{Pehlivan:2014zua,Balaji:2019fxd,Balaji:2020oig} between CP violation in the neutrino sector and the anomalous magnetic moment (AMM) of the neutrino.

Amplified CP violation is already expected in matter~\citep{Harrison:1999df} such as when travelling through the Earth's crust due to the abundance of electrons and the lack of muons and taus which preferentially affect the electron neutrinos via the weak interaction. This effect may also play a role in the primordial universe where neutrinos would propagate through incredibly matter dense gasses such as during the electron-positron epoch~\citep{Rafelski:2023emw}. Matter modifies CP violation because it changes the effective Hamiltonian of the propagating neutrinos, therefore we should look to other possible sources for modification of the Hamiltonian for CP violation amplification or suppression.

The source of direct CP violation in the neutrinos is ultimately attributable to the mismatch between the mass matrices of the charged flavors $(e,\mu,\tau)$ and the neutral flavors $(\nu_{e},\nu_{\mu},\nu_{\tau})$. The situation is analogous to the quark sector, where instead the relation is between the upper $(u,c,t)$ and lower quark $(d,s,b)$ flavors.

Therefore mass matrix for charged leptons does not commute with the mass matrix of the neutral leptons and cannot be simultaneously diagonalized except for special cases or degeneracy among the mass eigenstates. We can characterize CP violation by introducing the mixing matrices $V_{\ell k}$ (following the notation in \rchap{chap:neutrino}) which diagonalize the individual mass-matrices $M_{\ell\ell'}$ as follows
\begin{alignat}{1}
	\label{diagj:1}
    V_{\ell l}^{\dag}(M^{\nu}M^{\nu\dag})_{\ell\ell'}V_{\ell'k'} = D_{\ell\ell'}^{\nu} = \mathrm{diag}(m_{1}^{2},m_{2}^{2},m_{3}^{2})\,,\\
    \label{diagj:2}
    D_{\ell\ell'} = \mathrm{diag}(m_{e},m_{\mu},m_{\tau})\,,
\end{alignat}
We have specifically defined the charged leptons flavor states as being simultaneously mass eigenstates without a loss of generality. We will not consider oscillation among the charged leptons, though that may be an avenue of further study~\cite{Akhmedov:2007fk}.

%%%%%%%%%%%%%%%%%%%%%%%%%%%%%%%%%%%%%%%
\subsection{Jarlskog invariant}
\label{sec:jscalar}
%%%%%%%%%%%%%%%%%%%%%%%%%%%%%%%%%%%%%%%
The intrinsic CP violation inherent to these mass matrices can be described using the Jarlskog invariant $J$. We first define the commutator of the charged and neutral lepton mass matrices as 
\begin{alignat}{1}
	\label{comm:1} [M^{\nu}M^{\nu\dag},D^{2}]_{\ell\ell'} = C_{\ell\ell'}\,.
\end{alignat}
As the elements of the neutrino mixing matrix $V_{\ell k}$ can be experimentally determined, the size of the commutator can be expressed using~\req{diagj:1} and~\req{diagj:2} in terms of the mass eigenstates of the neutrino mass matrix
\begin{alignat}{1}
	\label{comm:2} [V(D^{\nu})^{2}V^{\dag},D^{2}]_{\ell\ell'} = C_{\ell\ell'}\,.
\end{alignat}
The matrix $C_{\ell\ell'}$ can be unwieldy, so following the procedure by~\citep{Jarlskog:1985ht,Jarlskog:1985cw,Jarlskog:2004be}, we take the determinant of $C_{\ell\ell'}$ in~\req{comm:2} which extracts the invariant quantity associated with the size of the CP violation present. Specifically we are interested in the imaginary portion given by
\begin{alignat}{1}
	\label{det:1} \mathrm{Im}\left[\mathrm{det}(C_{\ell\ell'})\right]=2\left(\Delta_{12}\Delta_{23}\Delta_{13}\right)\left(\Delta_{e\mu}\Delta_{\mu\tau}\Delta_{e\tau}\right)J\,,
\end{alignat}
where $J$ is the invariant quantity of interest. We define $\Delta_{ij}$ via the eigenstates of the mass matrices as
\begin{alignat}{1}
	\label{delta:1} \Delta_{ij}\equiv m^{2}_{i}-m^{2}_{j}\,.
\end{alignat}
We can also define the real portion of the determinant and define two quantities $R$ and $J$ together. These two scalars are written in terms of the components of the mixing matrix $V_{\ell k}$ as
\begin{alignat}{1}
	\label{j:1}
    {\cal J}_{ikjl} = \mathrm{Im}\left[V_{ij}V_{kl}V^{*}_{il}V^{*}_{kj}\right]=J\sum_{m,n}\epsilon_{ikm}\epsilon_{jln}\,,\\
    \label{j:2}
    {\cal R}_{ikjl} = \mathrm{Re}\left[V_{ij}V_{kl}V^{*}_{il}V^{*}_{kj}\right]=R\sum_{m,n}\epsilon_{ikm}\epsilon_{jln}\,.
\end{alignat}
The benefit of $J$ is it captures the `size' of CP violation in a single value and is identically zero for systems which preserve charge-parity symmetry. From \req{det:1} we can see that CP violation vanishes if the commutator of the mass matrices is purely real, and thus the mixing matrix is also purely real, or if there is degeneracy among the mass eigenstates which absorbs a degree of freedom.

While generally CP violation comes exclusively from the presence of three flavor generations in the Standard Model, we would like to expand the usage of the J-invariant to encompass a family of effects via modification of the mass matrix as was done in \rchap{chap:neutrino}. This would encompass CP violation from the number of flavor generations, physical systems (matter, strong fields, etc...) which effectively break CP, or in CP violating terms in the Lagrangian.

%%%%%%%%%%%%%%%%%%%%%%%%%%%%%%%%%%%%%%%
\subsection{Toy model: CP violation amplification}
\label{sec:amp}
%%%%%%%%%%%%%%%%%%%%%%%%%%%%%%%%%%%%%%%
\noindent While the specific change to neutrino mixing and CP violation depends on the model of the neutrino dipole moment, a demonstrative model would be to assume that the magnetic moment matrix was simply proportional to the natural mass matrix of the neutrino at low order. Therefore we substitute
\begin{alignat}{1}
	\label{simplek:1} M_{\ell\ell'}^{\nu}\rightarrow
    (M_{\ell\ell'}^{\nu})'=M_{\ell\ell'}^{\nu}(1+\kappa\sigma_{\alpha\beta}F^{\alpha\beta})\,.
\end{alignat}
If the same mechanism which produced neutrino masses also produced their dipoles through some BSM physics, this would not be an unreasonable assumption.

As the modified mass matrix commutes entirely with the original mass matrix, the mathematical structure of the commutator in \req{comm:1} is unchanged
\begin{alignat}{1}
	\label{commutes:1} \left[(M^{\nu})',M^{\nu}\right]=0\,\rightarrow
    \mathrm{Im}[\mathrm{det}(C_{\ell\ell'})] = \mathrm{Im}[\mathrm{det}(C_{\ell\ell'}')]\,.
\end{alignat}
The determinant calculation is then identical between the two mass matrices. While the overall determinant is fixed, the individual elements which make up the determinant are modified. This yields
\begin{alignat}{1}
	\label{det:0} \mathrm{Im}\left[\mathrm{det}(C_{\ell\ell'})\right] &= \left(\Delta_{12}\Delta_{23}\Delta_{13}\right)\left(\Delta_{e\mu}\Delta_{\mu\tau}\Delta_{e\tau}\right)J\,,\\
	\label{det:2} \mathrm{Im}\left[\mathrm{det}(C_{\ell\ell'}')\right] &= \left(\Delta_{12}'\Delta_{23}'\Delta_{13}'\right)\left(\Delta_{e\mu}\Delta_{\mu\tau}\Delta_{e\tau}\right)J'\,,
\end{alignat}
where we have added primes to denote new values due to remixing. A similar result occurs in matter in~\cite{Harrison:1999df}. The ratio of modified $J'$ to $J$ is the amplification (or suppression) of CP violation given by
\begin{alignat}{1}
	\label{ampj:1} \mathcal{R} = \frac{J'}{J} = \frac{\left(\Delta_{12}\Delta_{23}\Delta_{13}\right)}{\left(\Delta_{12}'\Delta_{23}'\Delta_{13}'\right)}\,.
\end{alignat}
In our simple proportionality model, this simplifies to
\begin{alignat}{1}
	\label{ampj:2} \mathcal{R} = \frac{1}{\left(1+\kappa\sigma_{\alpha\beta}F^{\alpha\beta}\right)^{6}}\,.
\end{alignat}
To first order, and evaluating $\sigma_{\alpha\beta}F^{\alpha\beta}$ for homogeneous magnetic fields, \req{ampj:2} can be expressed as
\begin{alignat}{1}
	\label{amp:3} \mathcal{R} = 1\pm12\kappa B+\mathcal{O}(B^{2})\,.
\end{alignat}
where we allow for aligned and anti-aligned spin states.

%%%%%%%%%%%%%%%%%%%%%%%%%%%%%%%%%%%%%%%
\subsection{Electric dipole moments and CP symmetry}
\label{sec:edm}
%%%%%%%%%%%%%%%%%%%%%%%%%%%%%%%%%%%%%%%
Here we state the standard picture of CP violation through an electric dipole. The structure of the Pauli term in \req{lamm:1} informs us how to construct the relativistic electric dipole moment (EDM); see~\cite{Knecht:2003kc,Jegerlehner:2017gek}. The generalization to include the electric dipole is
\begin{alignat}{1}
	\label{edm:1} \delta\mu\rightarrow\delta\tilde{\mu}\equiv\delta\mu+i\epsilon\gamma^{5}\,,
\end{alignat}
where $\epsilon$ is the EDM of the particle. As the natural electric dipole within the Dirac equation is zero, the presence of $\epsilon$ is always considered anomalous. The EDM Pauli Lagrangian term is
\begin{gather}
    \label{ledm:1}
    \mathcal{L}_\mathrm{EDM} = -{\bar\psi}\left(i\epsilon\gamma^{5}\frac{1}{2}\sigma_{\alpha\beta}F^{\alpha\beta}\right)\psi\,,
\end{gather}
which is of interest because of the inclusion of $\gamma^{5}$. Taking advantage of the properties of $\gamma^{5}$, we can write the EDM in \req{ledm:1} as
\begin{gather}
    \label{ledm:3}
    \gamma^{5}\sigma_{\mu\nu}=\frac{i}{2}\varepsilon_{\mu\nu\alpha\beta}\sigma^{\alpha\beta}\,\rightarrow
    \mathcal{L}_\mathrm{EDM} = +{\bar\psi}\left(\epsilon\frac{1}{2}\sigma^{\alpha\beta}F_{\alpha\beta}^{*}\right)\psi\,.
\end{gather}
which is more closely analogous to the structure of the AMM in \req{lamm:1} making use of the dual form of the electromagnetic field tensor shown in \req{em:2}.

Following a procedure similar to the one found in \rsec{sec:mom}, \req{ledm:3} reduces in the non-relativistic limit to the Hamiltonian density EDM interaction
\begin{gather}
    \label{ledm:4}
    \mathcal{H}_\mathrm{EDM} \approx \epsilon\chi^{\dag}\bb{\sigma}\cdot\bb{E}\chi\,.
\end{gather}
The electric dipole is important because the presence of one would signify charge-parity (CP) violation in the theory which provides a method to distinguish between matter and antimatter. We discuss briefly using the parity (P) and time (T) symmetries of the relevant spin $\bb{s}$ and field vectors $(\bb{E},\bb{B})$ and their inner products. The even and odd symmetries of each are printed in \rt{fig:cp}.

%%%%%%%%%%%%%%%%%%%%%%%%%%%%%%%%%%%%%%%
\begin{table}[ht]
 \centering
 \begin{tabular}{ r|c|c|c|c|c| }
 \multicolumn{1}{r}{}
 & \multicolumn{1}{c}{$\bb{E}$}
 & \multicolumn{1}{c}{$\bb{B}$}
 & \multicolumn{1}{c}{$\bb{s}$}
 & \multicolumn{1}{c}{$\bb{s}\!\cdot\!\bb{E}$}
 & \multicolumn{1}{c}{$\bb{s}\!\cdot\!\bb{B}$} \\
 \cline{2-6}
 \begin{tabular}[x]{@{}c@{}}T symmetry\\ $(t\rightarrow-t)$\end{tabular} & \textbf{even} & odd & odd & odd & \textbf{even} \TBstrut\\
 \cline{2-6}
 \begin{tabular}[x]{@{}c@{}}P symmetry\\ $(\bb{x}\rightarrow-\bb{x})$\end{tabular} & odd & \textbf{even} & \textbf{even} & odd & \textbf{even} \TBstrut\\
 \cline{2-6}
 \begin{tabular}[x]{@{}c@{}}PT symmetry\\ $(x^{\alpha}\rightarrow-x^{\alpha})$\end{tabular} & odd & odd & odd & \textbf{even} & \textbf{even} \TBstrut\\
 \cline{2-6}
 \end{tabular}\\ \,\Bstrut\\
 \caption{Time (T), parity (P) and PT symmetries of electric $\bb{E}$, magnetic $\bb{B}$, spin $\bb{s}$ three-vectors and the inner products which describe the dipole Hamiltonian terms.}
 \label{fig:cp}
\end{table}
%%%%%%%%%%%%%%%%%%%%%%%%%%%%%%%%%%%%%%%

The EDM term \req{ledm:4} is overall T-odd and P-odd while the magnetic dipole is T-even and P-even. While EDM dipoles are common in molecular systems, no electric dipole has ever been measured for an elementary particle nor composite particles like the proton or neutron despite extensive searching. As a point of comparison, the EDM of the electron is excluded~\citep{ACME:2018yjb,Roussy:2022cmp} by a bound of $|\epsilon_{e}/c|<4.1\times10^{-30}\, e\,\mathrm{cm}$.

For CPT symmetry to hold, \rt{fig:cp} implies that both the electric and magnetic dipoles must be C-even. C-symmetry is a more complicated concept to discuss as it is only well-defined relativistically where particle and antiparticle states are simultaneously described by the theory. The Dirac spinor charge conjugates as
\begin{align}
    \label{c:1}
    C:\psi\rightarrow\psi_{c}=\eta_{c}C(\bar\psi)^\mathrm{T}= \eta_{c}C\gamma_{0}^\mathrm{T}\psi^{*}\,,
\end{align}
where $C$ is the charge conjugation matrix satisfying the conjugation relation
\begin{align}
    \label{c:2}
    -C\gamma_{\alpha}^\mathrm{T}C^{-1}=\gamma_{\alpha}\,,
\end{align}
and $\eta_{c}$ is an arbitrary complex phase. The exact matrix expression of $C$ depends on the representational basis used (Dirac, Weyl, Majorana, etc...). We're specifically interesting in the following conjugations
\begin{align}
    \label{c:3}
    C:\bar\psi\gamma_{\alpha}\psi\rightarrow-\bar\psi\gamma_{\alpha}\psi\,,\qquad
    C:\bar\psi\sigma_{\alpha\beta}\psi\rightarrow-\bar\psi\sigma_{\alpha\beta}\psi\,,\qquad
    C:A^{\alpha}\rightarrow-A^{\alpha}\,.
\end{align}
As the spin density $\bar\psi\sigma_{\alpha\beta}\psi$ and vector potential $A^{\alpha}$ (and thus $F^{\alpha\beta}$) are both odd under charge conjugation, the combination present in the AMM Lagrangian \req{lamm:1} is C-even under charge conjugation. The same is true of the EDM Lagrangian \req{ledm:3} which is more easily seen when cast in terms of the dual tensor.

We note that the fields $\psi$, $\bar\psi$ and $A^{\alpha}$ are considered dynamical \emph{operators} within a quantum field theory. C-symmetry is therefore broken when considering an externally fixed background field such as $A_\mathrm{ext}^{\alpha}(x)$ as $C:A_\mathrm{ext}^{\alpha}(x)\rightarrow A_\mathrm{ext}^{\alpha}(x)$. This distinction bears some importance when discussing the neutrino flavor rotation in \rchap{chap:neutrino}. There is also interest in EDM behavior in the background of curved spacetime~\citep{Filho:2023lqe}.

%%%%%%%%%%%%%%%%%%%%%%%%%%%%%%%%%%%%%%%
\section{Spin in 5D Kaluza-Klein theory}
\label{sec:kk}
%%%%%%%%%%%%%%%%%%%%%%%%%%%%%%%%%%%%%%%
The `miracle' of Kaluza-Klein~\citep{theodor1921unitatsproblem,klein1926quantentheorie} is that both 4D gravitation and electromagnetism emerge from a higher dimension 5D gravitational theory. While the specific importance of Kaluza's result to the physical world (if there is one) has yet to be revealed, the ideas of Kaluza-Klein have been extensively used to showcase the emergence of unified physics from higher dimensional geometries~\citep{Ortin:2015hya}.

In the modern context, we understand that Kaluza-Klein in its original form cannot strictly be correct as we are operating purely in the classical regime and we are missing the incorporation of the weak and strong interactions as well. With all that said, there is still value in exploring the implications of Kaluza-Klein in that Kaluza's miracle might very well not be an accident, but a natural result from a fuller more complete implementation of these ideas~\citep{Overduin:1997sri}.

The main goal of this research effort is to obtain the spin dynamics of a test particle in the context of a 5D Kaluza-Klein style theory in the same spirit as was accomplished for the general relativistic torque equations
\begin{alignat}{1}
  \label{STRESS06} \frac{Dp^{\mu}}{D\tau}+\frac{1}{2}u^{\nu}s^{\rho\sigma}R^{\mu}_{\ \nu\rho\sigma}=0\,,\qquad
  \frac{Ds^{\mu\nu}}{D\tau}+2u^{[\mu}p^{\nu]}=0\,,
\end{alignat}
which couples spin and precession to a curved spacetime. An additional term is also required to accommodate spin in the kinematic equation. \req{STRESS06} is known as the Mathisson-Papapetrou-Dixon (MPD) equations~\citep{mathisson1937neue,Papapetrou:1951pa,Dixon:1970zza}. The rank-four tensor $R^{\mu}_{\ \nu\rho\sigma}$ is the Rienmann curvature defined by
\begin{alignat}{1}
  \label{STRESS08} R^{\mu}_{\ \nu\rho\sigma}V^{\nu}=2\nabla_{[\rho}\nabla_{\sigma]}V^{\mu}\,,
\end{alignat}
where $V^{\mu}$ is any arbitrary vector and the notation $[\rho,\sigma]$ indicates a commutator of indices. The significance of \req{STRESS06} is that particle motion will deviate away from traditional geodesic motion due to a coupling of the curvature to the spin.

To promote the above equations into the Kaluza-Klein framework, we consider Lorentz five-vectors $\hat{P}^{A}$ which are dressed with a hat and capital Latin letters indices $A\in(0-3,5)$. The position five-vector is then denoted by $\hat{x}^{A}=(x^{\mu},x_{5})$ where $x_{5}$ is the fifth coordinate position. The 5D Einstein-Hilbert action is
\begin{alignat}{1}
	\label{KALUZA01} \mathcal{S}[\hat{g}_{AB}]=\frac{c^{4}}{16\pi\hat{G}}\int\mathrm{d}\hat{x}^{5}\sqrt{-\hat{g}}\hat{R}\,.
\end{alignat}
The variable $\hat{G}$ is the 5D gravitational constant and $\hat{R}$ is the Ricci scalar in the 5D space-time. Kaluza-Klein theories are usually expressed as Ricci-flat theories with the scalar curvature $\hat{R}=0$, but in general nothing prevents us from including matter or other fields in the five-dimensional bulk by adding terms to the above action.

In a 5D spacetime, a particle under free-fall motion manifests as accelerated motion analogous to \req{STRESS06} caused by the electromagnetic force and an additional scalar force in the four-dimensional sector. Because particles with spin deviate from geodesics in free-fall, there should ultimately be spin precession generated by electromagnetism, the scalar field, and gravitation.

Therefore we propose a classical spin five-vector
\begin{alignat}{1}
	\label{KALUZA01a} \hat{s}^{A}=(s^{\alpha},s_{5})\,,
\end{alignat}
which relates the classical four-spin \req{fourspin} discussed in \rsec{sec:cspin} to a new fifth component of spin $s_{5}$. We note that analogously the fifth component of five-momentum is related to the mass of the particle; it is our suggestion that the fifth component of spin is related to the invariant spin magnitude. It is the study of this object that will be left to future publications.