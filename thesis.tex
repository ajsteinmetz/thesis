\documentclass[a4paper]{report}
% Leave uncommented if the LaTeX file is uploaded to arXiv.org
\pdfoutput=1
\pdfminorversion=7

% Packages
\usepackage{arxiv}
\usepackage[colorlinks=true,linkcolor=cyan,citecolor=cyan]{hyperref}
\usepackage[numbers]{natbib}
\usepackage{authblk}
\usepackage{caption}
\usepackage{subcaption}
\usepackage{graphicx}
\usepackage{amsmath}
\usepackage{amssymb}
\usepackage{epstopdf}
\usepackage{comment}
\usepackage{xcolor}
\usepackage{float}
\usepackage{doi}

% Useful macros for equations and units in HEP
\newcommand*{\TeV}{\text{ TeV}}
\newcommand*{\GeV}{\text{ GeV}}
\newcommand*{\MeV}{\text{ MeV}}
\newcommand*{\keV}{\text{ keV}}
\newcommand*{\eV}{\text{ eV}}
\newcommand*{\meV}{\text{ meV}}
\newcommand*{\bb}{\boldsymbol}
\newcommand*{\beqn}{\begin{equation}}
\newcommand*{\eeqn}{\end{equation}}
\newcommand{\req}[1]{Eq.\,(\ref{#1})}
\newcommand{\rf}[1]{Fig.~{\ref{#1}}}
\newcommand{\rsec}[1]{Sect.\,{\ref{#1}}}

% Useful macros for annotation
\newcommand*{\xred}{\color{red}}
\newcommand*{\xblue}{\color{blue}}
\newcommand*{\xgreen}{\color{green}}

\title{Thesis}

% Author Orcid ID: Define per author
\newcommand{\orcC}{0000-0001-5474-2649}

\author{Andrew Steinmetz\orc{\orcC}\thanks{Correspondence: \texttt{ajsteinmetz@arizona.edu}}
\\ Department of Physics, The University of Arizona, Tucson, AZ 85721, USA}

\begin{document}

\maketitle

\begin{abstract}
    {tbw}
\end{abstract}

\keywords{relativistic mechanics \and quantum mechanics \and magnetic moment \and magnetism}

%%%%%%%%%%%%%%%%%%%%%%%%%%%%%%%%%%%%%%%
\chapter{Overview and concepts}
    \section{Notation}
    \section{Spin}
        \subsection{Classical spin}
        \subsection{Quantum spin}
    \section{Magnetic and electric dipoles}
        \subsection{Anomalous magnetic moment}
    \section{Cosmology}
        \subsection{FLRW metric}
        \subsection{Conserved quantities under expansion}

%%%%%%%%%%%%%%%%%%%%%%%%%%%%%%%%%%%%%%%
\chapter{Classical magnetic dipole moments}
    \section{Stern-Gerlach force}
        \subsection{Amperian and Gilbert dipoles}
    \section{TBMT equations}
    \section{Magnetic spin potential}
        \subsection{Modified TBMT equations}
        \subsection{Unified Amperian and Gilbert dipoles}
        \subsection{Dynamic particle motion}
            \subsubsection{Charged particles}
            \subsubsection{Neutral particles}
    \section{Spin in 5D Kaluza-Klein model}
        \subsection{Correspondence to particle Lagrangians}

%%%%%%%%%%%%%%%%%%%%%%%%%%%%%%%%%%%%%%%
\chapter{Quantum magnetic dipole moments}
    \section{Schrodinger-Pauli equation}
    \section{Dirac and Dirac-Pauli equations}
        \subsection{Ehrenfest theorem for Stern-Gerlach forces}
    \section{Klein-Gordon-Pauli equation}
        \subsection{KGP in homogeneous fields}
        \subsection{KGP for hydrogen-like atoms}
            \subsubsection{Critical field strengths}
        \subsection{Improvements to KGP}
    \section{Extensions to Non-Albelian fields}

%%%%%%%%%%%%%%%%%%%%%%%%%%%%%%%%%%%%%%%
\chapter{Magnetization in primordial cosmology}
    \section{Electron-positron epoch of the universe}
        \subsection{Baryon content}
            \subsubsection{Entropy conservation}
        \subsection{Chemical fugacity}
        \subsection{Spin fugacity}
            \subsubsection{Non-relativistic spin fugacity}
    \section{Magnetized partition function}
        \subsection{Magnetized chemical potential}
        \subsection{Magnetization of the medium}
    \section{Temperature and density effects}

%%%%%%%%%%%%%%%%%%%%%%%%%%%%%%%%%%%%%%%
\chapter{Neutrinos and magnetism}
    \section{Neutrino masses}
        \subsection{Mass hierarchy}
        \subsection{Dirac neutrinos}
        \subsection{Majorana neutrinos}
            \subsubsection{See-saw mechanism}
    \section{Neutrino magnetic moments}
        \subsection{Direct moments}
        \subsection{Transition moments}
    \section{Flavor rotation}
        \subsection{PMNS matrix}
        \subsection{Magnetically induced rotation}
    \section{CP violation}
        \subsection{Jarlskog invariant}

\end{document}
